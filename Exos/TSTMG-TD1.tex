% Chargement des paquets

\usepackage{amsmath}
\usepackage{amsthm}
\usepackage{amsfonts}
\usepackage{amssymb}
\usepackage{enumerate}
\usepackage{mathtools}
\usepackage{bbm}
\usepackage{xparse, etoolbox}
\usepackage{enumerate}
\usepackage{mathabx}
\usepackage{minted}
\usepackage[french]{babel}
\usepackage{keytheorems}
\usepackage[theorems]{tcolorbox}
\usepackage{hyperref}

% Environnement

%Chargement des paquets
\usepackage{amsmath}
\usepackage{amsthm}
\usepackage{amsfonts}
\usepackage{amssymb}
\usepackage{enumerate}
\usepackage{mathtools}
\usepackage{bbm}
\usepackage{xparse, etoolbox}
\usepackage{enumerate}
\usepackage{mathabx}
\usepackage{minted}
\usepackage[french]{babel}
\usepackage{keytheorems}
\usepackage[theorems]{tcolorbox}
\usepackage{hyperref}
\usepackage[framemethod=tikz]{mdframed}

%Ensembles de nombres
\newcommand{\N} {\mathbb{N}}
\newcommand{\Ne}{\N^\ast}
\newcommand{\Z} {\mathbb{Z}}
\newcommand{\D} {\mathbb{D}}
\newcommand{\Q} {\mathbb{Q}}
\newcommand{\R} {\mathbb{R}}
\newcommand{\Rb}{\overline{\mathbb{R}}}
\newcommand{\Rp}{\R_+}
\newcommand{\Rm}{\R_-}
\newcommand{\K} {\mathbb{K}}
\newcommand{\Cx}{\mathbb{C}}

%Opérateurs
\newcommand{\equi}{\Leftrightarrow}

%Normes
\DeclarePairedDelimiter\abs{\lvert}{\rvert}

%tikz
\usepackage{tikz, pgfplots}
\usetikzlibrary{positioning}
\usetikzlibrary{shapes.geometric}
\usetikzlibrary{positioning}
\usetikzlibrary {angles}
\usepackage{tkz-euclide}

\tikzset{
dot/.style = {circle, fill=#1, minimum size=5pt,
              inner sep=0pt, outer sep=0pt},
dot/.default = black % size of the circle diameter
}

 % for braces
\usetikzlibrary{decorations.pathreplacing}
% for hashing area
\usetikzlibrary{patterns}
% tableaux var, signe
% source https://www.sqlpac.com/fr/documents/latex-package-tkz-tab-tikz-tableaux-de-signes-et-de-variations-de-fonctions.html
\usepackage{tkz-tab}

\tikzset{
	every node/.style = {font=\Large}
}

\tikzset{
	every axis/.style = {clip=true, grid style = {opacity=.5}}
}

%Interface théorème
\renewcommand*{\proofname}{Démonstration}

\theoremstyle{definition}
\newtheorem*{nota}{Notation}

\theoremstyle{definition}
\newtheorem*{conv}{Convention}

\theoremstyle{definition}
\newtheorem{ex}{Exemple}[section]

\theoremstyle{remark}
\newtheorem{rmq}{Remarque}

\theoremstyle{definition}
\newtheorem*{idea}{Idée}

%styles pour théorèmes
\newkeytheoremstyle{tcb-thm}
  {
    headpunct={},
    notebraces={}{},
    noteseparator={ : },
    notefont=\bfseries,
    bodyfont=\slshape,
    tcolorbox={
      arc=0mm,
      colback=blue!5!white,
      colframe=blue!50!black,
      },
  }
  
\newkeytheorem{thm}[
  name=Théorème,
  parent=section,
  style=tcb-thm,
  ]  
  
\newkeytheoremstyle{tcb-prop}
  {
    headpunct={},
    notebraces={}{},
    noteseparator={ : },
    notefont=\bfseries,
    bodyfont=\slshape,
    tcolorbox={
      arc=0mm,
      colback=blue!5!white,
      colframe=blue!75!black,
      },
  }
 
\newkeytheorem{prop}[
  name=Proposition,
  parent=section,
  style=tcb-prop,
  ] 
  
\newkeytheorem{coro}[
  name=Corollaire,
  parent=section,
  style=tcb-prop,
  ]
  
\newkeytheoremstyle{tcb-lem}
  {
    headpunct={},
    notebraces={}{},
    noteseparator={ : },
    notefont=\bfseries,
    bodyfont=\slshape,
    tcolorbox={
      arc=0mm,
      colback=blue!5!white,
      colframe=blue!100!black,
      },
  }
  
\newkeytheorem{lem}[
  name=Lemme,
  parent=section,
  style=tcb-lem,
  ]
 
  
\newkeytheoremstyle{tcb-def}
  {
    headpunct={},
    notebraces={}{},
    noteseparator={ : },
    notefont=\bfseries,
    bodyfont=\slshape,
    tcolorbox={
      arc=0mm,
      colback=orange!5!white,
      colframe=orange!75!black,
      },
  }
  
\newkeytheorem{defn}[
  name=Définition,
  parent=section,
  style=tcb-def,
  ]
 
\newkeytheorem{rapl}[
  name=Rappel,
  style=tcb-def,
  ]
  
\newkeytheoremstyle{tcb-exo}
  {
    headpunct={},
    notebraces={}{},
    noteseparator={ : },
    notefont=\bfseries,
    bodyfont=\slshape,
    tcolorbox={
      arc=0mm,
      colframe=black,
      colback=white,
      },
  }
  
\newkeytheorem{exo}[
  name=Exercice,
  style=tcb-exo,
  ]
  
\surroundwithmdframed[
	hidealllines=true,
	leftline=true,
	innerleftmargin=10pt,
	innerrightmargin=10pt,
	innertopmargin=-4pt,
	nobreak=true,
]{proof}

% Commandes perso

%Ensembles de nombres
\newcommand{\N} {\mathbb{N}}
\newcommand{\Ne}{\N^\ast}
\newcommand{\Z} {\mathbb{Z}}
\newcommand{\D} {\mathbb{D}}
\newcommand{\Q} {\mathbb{Q}}
\newcommand{\R} {\mathbb{R}}
\newcommand{\Rb}{\overline{\mathbb{R}}}
\newcommand{\Rp}{\R_+}
\newcommand{\Rm}{\R_-}
\newcommand{\K} {\mathbb{K}}
\newcommand{\Cx}{\mathbb{C}}

%Opérateurs
\newcommand{\equi}{\Leftrightarrow}

%Normes
\DeclarePairedDelimiter\abs{\lvert}{\rvert}

%Commande d'exo
\newcommand{\exe}[4]{
	\begin{Exercise}[title=#1, label=#3]
		\marginpar{\mbox{\scriptsize(solution p.\pageref{\ExerciseLabel-Answer})}}
		#2
	\end{Exercise}
	\begin{Answer}[ref=#3]
		#4
	\end{Answer}
}

%tikz
\usepackage{tikz, pgfplots}
\usetikzlibrary{positioning}
\usetikzlibrary{shapes.geometric}
\usetikzlibrary{positioning}
\usetikzlibrary {angles}
\usepackage{tkz-euclide}

\tikzset{
dot/.style = {circle, fill=#1, minimum size=5pt,
              inner sep=0pt, outer sep=0pt},
dot/.default = black % size of the circle diameter
}

 % for braces
\usetikzlibrary{decorations.pathreplacing}
% for hashing area
\usetikzlibrary{patterns}
% tableaux var, signe
% source https://www.sqlpac.com/fr/documents/latex-package-tkz-tab-tikz-tableaux-de-signes-et-de-variations-de-fonctions.html
\usepackage{tkz-tab}

\tikzset{
	every node/.style = {font=\Large}
}

\tikzset{
	every axis/.style = {clip=true, grid style = {opacity=.5}}
}

\begin{document}

\pagestyle{fancy}
\fancyhead[L]{Seconde}
\fancyhead[C]{\textbf{TD n°1 : ensemble de nombres}}
\fancyhead[R]{\today}

La difficulté des exercices est dénotée par des étoiles (de 0 à 3).

\exe{, difficulty=0}{ 
Donner la définition explicite de la suite des nombres pairs. 
\end{enumerate}
}{exe:exo1}

\exe{Fractions, difficulty=0}{
Donner la définition par récurrence de la suite des nombres pairs
}{exe:exo2}

\exe{, difficulty=0}{
On considère la suite arithmétique de premier terme $u_0 = 2$ et de raison 3. Tracer les termes $u_0, u_1, \dots, u_5$ dans un repère orthonormé. Que constatez-vous ?
}{exe:exo3}

\exe{, difficulty=0}{ 
On considère la suite arithmétique de premier terme $u_0 = 3$ et de raison $2$. Tracer les termes $u_0, u_1, u_2, u_3$ dans un repère orthonormé. Que constatez-vous ?
}{exe:exo4}

\exe{, difficulty=1}{ 
Pour chaque affirmation ci-dessous, dire si elle est vraie ou fausse. Justifier.
\begin{enumerate}
\item La suite définie par $u_n = 3 \cdot n + 4$ pour $n \in \N$ est une suite arithmétique.
\item La suite définie par $u_1 = 4$ et $u_{n+1} = u_{n} + \dfrac14$ pour $n \in \N$ est une suite arithmétique.
\item Si $(u_n)$ est une suite arithmétique de raison $5$ et de premier terme $u_0 \in \R$ alors $u_{n+10} - u_n = 50$. 
\item Si $(u_n)$ est une suite arithmétique de raison $10^{-3}$ et de premier terme $u_0=0$, alors pour tout $n \in \N, u_n < 1$.
\end{enumerate}
}{exe:exo5}

\exe{, difficulty=3}{
\begin{enumerate}
\item On considère une suite arithmétique $(u_n)$ de premier terme $u_0$ et de raison $r > 0$. On cherche le plus petit rang $N \in N$ tel que $u_N \geq x$ où $x$ est un réel positif arbitraire.
\begin{enumerate}
\item Que vaut $u_1 - u_0$ ? $u_{25} - u_{20}$ ?
\item En déduire l'expression de $u_n - u_0$ puis conclure.
\item Proposer un algorithme (python ou language naturel) permettant de calculer le rang $N$.
\end{enumerate}
\item On considère maintenant une suite géométrique $(u_n)$ de premier terme $u_0$ et de raison $q > 0$. On cherche à nouveau le plus petit rang $N \in \N$ tel que $u_N \geq x$ où $x$ est un réel positif arbitraire.
\begin{enumerate}
\item Que vaut $\frac{u_1}{u_0}$ ? $\frac{u_{25}}{u_{20}}$ ?
\item En déduire l'expression de $\frac{u_n}{u_0}$.
\item On est maintenant face à une difficulté technique (on ne connaît pas la fonction réciproque de la fonction ``puissance $n$''). On va donc devoir faire une recherche ``à la main'' du seuil $N$. Proposer un algorithme effectuant cette recherche.
\end{enumerate}
\end{enumerate}
}{exe:exo5}

\exe{, difficulty=1}{
Milon de Crotone était un athléte de la Grèce antique (né aux alentours de 550 avant Jésus Christ). Un mythe à son propos explique que, pour son entraînement, il décida de soulever un jeune veau tous les jours. Le veau grandissant, la charge soulevée par Milon augmentait progressivement. Lorsque le veau devint adulte, Milon pouvait toujours le soulever, par son entraînement il avait acquis, petit à petit, une force Herculéenne.
\begin{enumerate}
\item On suppose que le veau de Milon pesait 40 kg à sa naissance et gagnait 1 kg chaque jour. Quel outil mathématiques permet de modéliser la croissance du veau ? Expliquer votre choix.
\item Milon s'entraîne ainsi durant une année entière. Quel charge est-il capable de soulever à l'issue de son entraînement ?
\item Quelle charge cumulée (i.e. la somme des charges journalières) Milon a-t-il soulevé durant cette année d'entraînement ?
\end{enumerate}
}{exe:exo6}

\exe{, difficulty=1}{
Le taux de reproduction du premier variant du SARS-COV-2 était d'environ 3, ce qui signifie qu'une personne contaminée infectait en moyenne 3 nouvelles personnes. On suppose que le cluster souche (l'ensemble des premiers infectés) contenait dix individus.
\begin{enumerate}
\item Quel outil mathématiques permet de modéliser la propagation du virus dans la population ? Expliquer votre choix.
\item On supposera qu'un individu contaminé infecte 3 nouveaux individus en un jour. Proposer une estimation du nombre de contaminés après 100 jours.
\item On estime la population française à 70 millions d'habitant. En reprenant les hypothèses de la question précédente, estimer le nombre de jours nécessaire à la transmission du virus à l'intégralité de la population.
\end{enumerate}
}{exe:exo7}

\exe{, difficulty=2}{
Supposons qu'une oportunité financière permette de rémunérer un placement de capital à hauteur de $x \%$ (avec $x$ un réel positif).
Le capital ainsi placé raportera donc $x \%$ d'intérêt après un temps $t$ (par exemple, $2 \%$ tous les ans). Le principe des intérêts
composés est de réinvestir systématiquement les gains du capital à chaque versement des intérêts, ainsi, les intérêts de la période
suivante ne porteront pas uniquement sur le capital de départ mais sur ce dernier additionné des intérêts précédemment générés. 
\begin{enumerate}
\item Pour étudier le principe des intérêts composés, on considère un capital initial noté $u_0$ et une rémunération notée $q$ ($q$ n'est pas exprimé en pourcentage, il correspond à $1+x$, par exemple une rémunération de $2 \%$ correspond à une multiplication par $1,02$).
\begin{enumerate} 
\item On note $u_1$ le capital total (i.e. le capital initial additionné des intérêts perçus) après un versement d'intérêt. Exprimer $u_1$ en fonction de $u_0$ et de $q$.
\item On suppose que les intérêts générés sont systématiquement réinvestis, exprimer $u_2$ en fonction de $u_1$. En déduire une expression par récurrence de $u_{n+1}$ en fonction de $u_n$ pour $n \in \N$.
\item Quel outil mathématiques permet de simuler des intérêts composés ?
\item On dispose de $1000 \texteuro$ que l'on investit sur les marchés financiers. Ce placement est rémunéré à hauteur de $2 \%$ tous les ans. Serons nous milliardaire en moins de 10 ans ?
\end{enumerate}
\item On suppose maintenant que l'investisseur est en capacité de verser régulièrement un capital donné (par exemple $200 \texteuro$). Ainsi, au capital initial viendront non seulement s'additioner les intérêts perçus mais aussi des versements réguliers.
\begin{enumerate}
\item On note $u_0$ le capital initial, $c$ la capacité d'investissement de l'investisseur (i.e. le montant de ses versements réguliers) et $q$ la rémunération du capital. Exprimer $u_1$ puis $u_2$ en fonction de $c$ et de $q$. \\
\item En déduire l'expression de $u_n$ en fonction de $c$, $q$ et $n$. Que remarquez-vous ? Utiliser le cours pour simplifier cette formule.
\end{enumerate}
\end{enumerate}
}{exe:exo8}

\end{document}