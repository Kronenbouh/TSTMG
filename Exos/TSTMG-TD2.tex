% Chargement des paquets

\usepackage{amsmath}
\usepackage{amsthm}
\usepackage{amsfonts}
\usepackage{amssymb}
\usepackage{enumerate}
\usepackage{mathtools}
\usepackage{bbm}
\usepackage{xparse, etoolbox}
\usepackage{enumerate}
\usepackage{mathabx}
\usepackage{minted}
\usepackage[french]{babel}
\usepackage{keytheorems}
\usepackage[theorems]{tcolorbox}
\usepackage{hyperref}

% Environnement

%Chargement des paquets
\usepackage{amsmath}
\usepackage{amsthm}
\usepackage{amsfonts}
\usepackage{amssymb}
\usepackage{enumerate}
\usepackage{mathtools}
\usepackage{bbm}
\usepackage{xparse, etoolbox}
\usepackage{enumerate}
\usepackage{mathabx}
\usepackage{minted}
\usepackage[french]{babel}
\usepackage{keytheorems}
\usepackage[theorems]{tcolorbox}
\usepackage{hyperref}
\usepackage[framemethod=tikz]{mdframed}

%Ensembles de nombres
\newcommand{\N} {\mathbb{N}}
\newcommand{\Ne}{\N^\ast}
\newcommand{\Z} {\mathbb{Z}}
\newcommand{\D} {\mathbb{D}}
\newcommand{\Q} {\mathbb{Q}}
\newcommand{\R} {\mathbb{R}}
\newcommand{\Rb}{\overline{\mathbb{R}}}
\newcommand{\Rp}{\R_+}
\newcommand{\Rm}{\R_-}
\newcommand{\K} {\mathbb{K}}
\newcommand{\Cx}{\mathbb{C}}

%Opérateurs
\newcommand{\equi}{\Leftrightarrow}

%Normes
\DeclarePairedDelimiter\abs{\lvert}{\rvert}

%tikz
\usepackage{tikz, pgfplots}
\usetikzlibrary{positioning}
\usetikzlibrary{shapes.geometric}
\usetikzlibrary{positioning}
\usetikzlibrary {angles}
\usepackage{tkz-euclide}

\tikzset{
dot/.style = {circle, fill=#1, minimum size=5pt,
              inner sep=0pt, outer sep=0pt},
dot/.default = black % size of the circle diameter
}

 % for braces
\usetikzlibrary{decorations.pathreplacing}
% for hashing area
\usetikzlibrary{patterns}
% tableaux var, signe
% source https://www.sqlpac.com/fr/documents/latex-package-tkz-tab-tikz-tableaux-de-signes-et-de-variations-de-fonctions.html
\usepackage{tkz-tab}

\tikzset{
	every node/.style = {font=\Large}
}

\tikzset{
	every axis/.style = {clip=true, grid style = {opacity=.5}}
}

%Interface théorème
\renewcommand*{\proofname}{Démonstration}

\theoremstyle{definition}
\newtheorem*{nota}{Notation}

\theoremstyle{definition}
\newtheorem*{conv}{Convention}

\theoremstyle{definition}
\newtheorem{ex}{Exemple}[section]

\theoremstyle{remark}
\newtheorem{rmq}{Remarque}

\theoremstyle{definition}
\newtheorem*{idea}{Idée}

%styles pour théorèmes
\newkeytheoremstyle{tcb-thm}
  {
    headpunct={},
    notebraces={}{},
    noteseparator={ : },
    notefont=\bfseries,
    bodyfont=\slshape,
    tcolorbox={
      arc=0mm,
      colback=blue!5!white,
      colframe=blue!50!black,
      },
  }
  
\newkeytheorem{thm}[
  name=Théorème,
  parent=section,
  style=tcb-thm,
  ]  
  
\newkeytheoremstyle{tcb-prop}
  {
    headpunct={},
    notebraces={}{},
    noteseparator={ : },
    notefont=\bfseries,
    bodyfont=\slshape,
    tcolorbox={
      arc=0mm,
      colback=blue!5!white,
      colframe=blue!75!black,
      },
  }
 
\newkeytheorem{prop}[
  name=Proposition,
  parent=section,
  style=tcb-prop,
  ] 
  
\newkeytheorem{coro}[
  name=Corollaire,
  parent=section,
  style=tcb-prop,
  ]
  
\newkeytheoremstyle{tcb-lem}
  {
    headpunct={},
    notebraces={}{},
    noteseparator={ : },
    notefont=\bfseries,
    bodyfont=\slshape,
    tcolorbox={
      arc=0mm,
      colback=blue!5!white,
      colframe=blue!100!black,
      },
  }
  
\newkeytheorem{lem}[
  name=Lemme,
  parent=section,
  style=tcb-lem,
  ]
 
  
\newkeytheoremstyle{tcb-def}
  {
    headpunct={},
    notebraces={}{},
    noteseparator={ : },
    notefont=\bfseries,
    bodyfont=\slshape,
    tcolorbox={
      arc=0mm,
      colback=orange!5!white,
      colframe=orange!75!black,
      },
  }
  
\newkeytheorem{defn}[
  name=Définition,
  parent=section,
  style=tcb-def,
  ]
 
\newkeytheorem{rapl}[
  name=Rappel,
  style=tcb-def,
  ]
  
\newkeytheoremstyle{tcb-exo}
  {
    headpunct={},
    notebraces={}{},
    noteseparator={ : },
    notefont=\bfseries,
    bodyfont=\slshape,
    tcolorbox={
      arc=0mm,
      colframe=black,
      colback=white,
      },
  }
  
\newkeytheorem{exo}[
  name=Exercice,
  style=tcb-exo,
  ]
  
\surroundwithmdframed[
	hidealllines=true,
	leftline=true,
	innerleftmargin=10pt,
	innerrightmargin=10pt,
	innertopmargin=-4pt,
	nobreak=true,
]{proof}

% Commandes perso

%Ensembles de nombres
\newcommand{\N} {\mathbb{N}}
\newcommand{\Ne}{\N^\ast}
\newcommand{\Z} {\mathbb{Z}}
\newcommand{\D} {\mathbb{D}}
\newcommand{\Q} {\mathbb{Q}}
\newcommand{\R} {\mathbb{R}}
\newcommand{\Rb}{\overline{\mathbb{R}}}
\newcommand{\Rp}{\R_+}
\newcommand{\Rm}{\R_-}
\newcommand{\K} {\mathbb{K}}
\newcommand{\Cx}{\mathbb{C}}

%Opérateurs
\newcommand{\equi}{\Leftrightarrow}

%Normes
\DeclarePairedDelimiter\abs{\lvert}{\rvert}

%Commande d'exo
\newcommand{\exe}[4]{
	\begin{Exercise}[title=#1, label=#3]
		\marginpar{\mbox{\scriptsize(solution p.\pageref{\ExerciseLabel-Answer})}}
		#2
	\end{Exercise}
	\begin{Answer}[ref=#3]
		#4
	\end{Answer}
}

%tikz
\usepackage{tikz, pgfplots}
\usetikzlibrary{positioning}
\usetikzlibrary{shapes.geometric}
\usetikzlibrary{positioning}
\usetikzlibrary {angles}
\usepackage{tkz-euclide}

\tikzset{
dot/.style = {circle, fill=#1, minimum size=5pt,
              inner sep=0pt, outer sep=0pt},
dot/.default = black % size of the circle diameter
}

 % for braces
\usetikzlibrary{decorations.pathreplacing}
% for hashing area
\usetikzlibrary{patterns}
% tableaux var, signe
% source https://www.sqlpac.com/fr/documents/latex-package-tkz-tab-tikz-tableaux-de-signes-et-de-variations-de-fonctions.html
\usepackage{tkz-tab}

\tikzset{
	every node/.style = {font=\Large}
}

\tikzset{
	every axis/.style = {clip=true, grid style = {opacity=.5}}
}
\usepackage{graphicx,wrapfig}
\usepackage{eurosym}
\usepackage{varwidth}
\graphicspath{ {./images/} }
\usetikzlibrary {datavisualization.formats.functions}

\begin{document}

\pagestyle{fancy}
\fancyhead[L]{Terminale STMG}
\fancyhead[C]{\textbf{TD n°2 : logarithme}}
\fancyhead[R]{\today}

La difficulté des exercices est dénotée par des étoiles (de 0 à 3).

\exe{}{
Expliquer, à l'aide du cours, pourquoi les égalités ci-dessous sont vraies :
\begin{multicols}{2}
\begin{enumerate}[label=(\alph*)]
\item $\log(24)= 3\log(2) + \log(3)$
\item $\log(5) = \log(35) - \log(7)$
\item $\log(1024) = 10 \log(2)$
\item $\log(25) = \log(5) - \log(20) + 2\log(10)$
\end{enumerate}
\end{multicols}
}{exe:1}{

}

\exe{, difficulty=0}{
Résoudre les équations et inéquations suivantes :
\begin{multicols}{3}
\begin{enumerate}[label=(\alph*)]
\item $8^x = 123$
\item $x^{63} = 4~812$
\item $5^x \times 7^x = 28$
\item $12^x < 854$
\item $x^{25} > 28~412$
\item $25 < 9^x \leq 89$
\end{enumerate}
\end{multicols}
}{exe:2}{

}

\exe{, difficulty=1}{ 
On considère une suite géométrique $(u_n)$ de premier terme $u_0=1$ et de raison $q=10$.
On voudrait représenter les points $(n, u_n)$ dans un repère orthonormé.
\begin{enumerate}
\item Calculer quelques valeurs de $u_n$. Remarquez-vous un problème ?
\item On propose de remplacer l'axe des ordonnées par un axe à \textbf{l'échelle logarithmique}. Pour cela, on remplace chaque graduation de l'axe par son logarithme.
\begin{enumerate}
\item rappeler les valeurs de $\log(1), \log(10)$ puis $(\log(10^n)$ pour $n \in \N$.
\item Tracer le nouveau repère avec l'axe à l'échelle logarithmique.
\item Placer les points $(n, u_n)$ pour $n$ allant de 1 à 5 dans ce repère. Que constatez-vous ?
\end{enumerate}
\end{enumerate}
\textit{Le logarithme ``transforme les multiplications en additions'', il transforme donc les évolutions exponentielles en évolutions linéaires.}
}{exe:3}{

}

\exe{, difficulty=2}{ 
On étudie une épidémie virale dont le taux de transmission est estimé à $2,7$ par semaine, 
ce qui signifie qu'un infecté transmet le virus à $2,7$ nouvelles personnes \textbf{en moyenne} après une semaine. 
Le cluster de départ est de 9 individus.
\begin{enumerate}
\item Proposer un outil mathématiques permettant de modéliser la transmission du virus.
\item Donner le nombre de semaines nécessaires à la contamination de la planète entière (i.e. 7 milliards d'individus).
\end{enumerate}
}{exe:4}{

}

\exe{, difficulty=2}{ 
On appelle ordre de grandeur d'un nombre $x$ la borne inférieure d'un encadrement de la forme :
\[ 10^n < x < 10^{n+1} \qquad \text{ avec } n \in \N. \]
En d'autres termes, l'ordre de grandeur d'un nombre donne sa valeur approchée en termes de puissance de 10 
(dans les milliers, dans les millions, dans les milliards, etc.).
\begin{enumerate}
\item Donner l'ordre de grandeur du nombre $3,6^{87}$.
\item Combien devrait-on utiliser de chiffres pour écrire (en entier) le nombre $3,6^{87}$ ?
\item Mêmes questions avec le nombre $11^{75}$.
\end{enumerate}
}{exe:5}{


}

\exe{, difficulty=3}{ 
On considère un nombre $a>0$ et les deux équations ci-dessous :
\begin{align}
7^x  & = a \\
10^x & = a
\end{align}
On donne $\log(7) \approx 0,85$.
\begin{enumerate}
\item
Sans effectuer de calculs, donner la valeur de $a$ pour laquelle les deux équations admettent la même solution. 
\item
Pour quelles valeurs de $a$ les solutions de l'équation (1) sont strictement supérieures à celles de l'équation (2) ? Justifier.
\end{enumerate}
}{exe:6}{

}

%\newpage
%\fancyhead[C]{\textbf{Solutions}}
%\shipoutAnswer

\end{document}