% Chargement des paquets

\usepackage{amsmath}
\usepackage{amsthm}
\usepackage{amsfonts}
\usepackage{amssymb}
\usepackage{enumerate}
\usepackage{mathtools}
\usepackage{bbm}
\usepackage{xparse, etoolbox}
\usepackage{enumerate}
\usepackage{mathabx}
\usepackage{minted}
\usepackage[french]{babel}
\usepackage{keytheorems}
\usepackage[theorems]{tcolorbox}
\usepackage{hyperref}

% Environnement

%Chargement des paquets
\usepackage{amsmath}
\usepackage{amsthm}
\usepackage{amsfonts}
\usepackage{amssymb}
\usepackage{enumerate}
\usepackage{mathtools}
\usepackage{bbm}
\usepackage{xparse, etoolbox}
\usepackage{enumerate}
\usepackage{mathabx}
\usepackage{minted}
\usepackage[french]{babel}
\usepackage{keytheorems}
\usepackage[theorems]{tcolorbox}
\usepackage{hyperref}
\usepackage[framemethod=tikz]{mdframed}

%Ensembles de nombres
\newcommand{\N} {\mathbb{N}}
\newcommand{\Ne}{\N^\ast}
\newcommand{\Z} {\mathbb{Z}}
\newcommand{\D} {\mathbb{D}}
\newcommand{\Q} {\mathbb{Q}}
\newcommand{\R} {\mathbb{R}}
\newcommand{\Rb}{\overline{\mathbb{R}}}
\newcommand{\Rp}{\R_+}
\newcommand{\Rm}{\R_-}
\newcommand{\K} {\mathbb{K}}
\newcommand{\Cx}{\mathbb{C}}

%Opérateurs
\newcommand{\equi}{\Leftrightarrow}

%Normes
\DeclarePairedDelimiter\abs{\lvert}{\rvert}

%tikz
\usepackage{tikz, pgfplots}
\usetikzlibrary{positioning}
\usetikzlibrary{shapes.geometric}
\usetikzlibrary{positioning}
\usetikzlibrary {angles}
\usepackage{tkz-euclide}

\tikzset{
dot/.style = {circle, fill=#1, minimum size=5pt,
              inner sep=0pt, outer sep=0pt},
dot/.default = black % size of the circle diameter
}

 % for braces
\usetikzlibrary{decorations.pathreplacing}
% for hashing area
\usetikzlibrary{patterns}
% tableaux var, signe
% source https://www.sqlpac.com/fr/documents/latex-package-tkz-tab-tikz-tableaux-de-signes-et-de-variations-de-fonctions.html
\usepackage{tkz-tab}

\tikzset{
	every node/.style = {font=\Large}
}

\tikzset{
	every axis/.style = {clip=true, grid style = {opacity=.5}}
}

%Interface théorème
\renewcommand*{\proofname}{Démonstration}

\theoremstyle{definition}
\newtheorem*{nota}{Notation}

\theoremstyle{definition}
\newtheorem*{conv}{Convention}

\theoremstyle{definition}
\newtheorem{ex}{Exemple}[section]

\theoremstyle{remark}
\newtheorem{rmq}{Remarque}

\theoremstyle{definition}
\newtheorem*{idea}{Idée}

%styles pour théorèmes
\newkeytheoremstyle{tcb-thm}
  {
    headpunct={},
    notebraces={}{},
    noteseparator={ : },
    notefont=\bfseries,
    bodyfont=\slshape,
    tcolorbox={
      arc=0mm,
      colback=blue!5!white,
      colframe=blue!50!black,
      },
  }
  
\newkeytheorem{thm}[
  name=Théorème,
  parent=section,
  style=tcb-thm,
  ]  
  
\newkeytheoremstyle{tcb-prop}
  {
    headpunct={},
    notebraces={}{},
    noteseparator={ : },
    notefont=\bfseries,
    bodyfont=\slshape,
    tcolorbox={
      arc=0mm,
      colback=blue!5!white,
      colframe=blue!75!black,
      },
  }
 
\newkeytheorem{prop}[
  name=Proposition,
  parent=section,
  style=tcb-prop,
  ] 
  
\newkeytheorem{coro}[
  name=Corollaire,
  parent=section,
  style=tcb-prop,
  ]
  
\newkeytheoremstyle{tcb-lem}
  {
    headpunct={},
    notebraces={}{},
    noteseparator={ : },
    notefont=\bfseries,
    bodyfont=\slshape,
    tcolorbox={
      arc=0mm,
      colback=blue!5!white,
      colframe=blue!100!black,
      },
  }
  
\newkeytheorem{lem}[
  name=Lemme,
  parent=section,
  style=tcb-lem,
  ]
 
  
\newkeytheoremstyle{tcb-def}
  {
    headpunct={},
    notebraces={}{},
    noteseparator={ : },
    notefont=\bfseries,
    bodyfont=\slshape,
    tcolorbox={
      arc=0mm,
      colback=orange!5!white,
      colframe=orange!75!black,
      },
  }
  
\newkeytheorem{defn}[
  name=Définition,
  parent=section,
  style=tcb-def,
  ]
 
\newkeytheorem{rapl}[
  name=Rappel,
  style=tcb-def,
  ]
  
\newkeytheoremstyle{tcb-exo}
  {
    headpunct={},
    notebraces={}{},
    noteseparator={ : },
    notefont=\bfseries,
    bodyfont=\slshape,
    tcolorbox={
      arc=0mm,
      colframe=black,
      colback=white,
      },
  }
  
\newkeytheorem{exo}[
  name=Exercice,
  style=tcb-exo,
  ]
  
\surroundwithmdframed[
	hidealllines=true,
	leftline=true,
	innerleftmargin=10pt,
	innerrightmargin=10pt,
	innertopmargin=-4pt,
	nobreak=true,
]{proof}

% Commandes perso

%Ensembles de nombres
\newcommand{\N} {\mathbb{N}}
\newcommand{\Ne}{\N^\ast}
\newcommand{\Z} {\mathbb{Z}}
\newcommand{\D} {\mathbb{D}}
\newcommand{\Q} {\mathbb{Q}}
\newcommand{\R} {\mathbb{R}}
\newcommand{\Rb}{\overline{\mathbb{R}}}
\newcommand{\Rp}{\R_+}
\newcommand{\Rm}{\R_-}
\newcommand{\K} {\mathbb{K}}
\newcommand{\Cx}{\mathbb{C}}

%Opérateurs
\newcommand{\equi}{\Leftrightarrow}

%Normes
\DeclarePairedDelimiter\abs{\lvert}{\rvert}

%Commande d'exo
\newcommand{\exe}[4]{
	\begin{Exercise}[title=#1, label=#3]
		\marginpar{\mbox{\scriptsize(solution p.\pageref{\ExerciseLabel-Answer})}}
		#2
	\end{Exercise}
	\begin{Answer}[ref=#3]
		#4
	\end{Answer}
}

%tikz
\usepackage{tikz, pgfplots}
\usetikzlibrary{positioning}
\usetikzlibrary{shapes.geometric}
\usetikzlibrary{positioning}
\usetikzlibrary {angles}
\usepackage{tkz-euclide}

\tikzset{
dot/.style = {circle, fill=#1, minimum size=5pt,
              inner sep=0pt, outer sep=0pt},
dot/.default = black % size of the circle diameter
}

 % for braces
\usetikzlibrary{decorations.pathreplacing}
% for hashing area
\usetikzlibrary{patterns}
% tableaux var, signe
% source https://www.sqlpac.com/fr/documents/latex-package-tkz-tab-tikz-tableaux-de-signes-et-de-variations-de-fonctions.html
\usepackage{tkz-tab}

\tikzset{
	every node/.style = {font=\Large}
}

\tikzset{
	every axis/.style = {clip=true, grid style = {opacity=.5}}
}
\usepackage{graphicx,wrapfig}
\usepackage{eurosym}
\usepackage{varwidth}
\graphicspath{ {./images/} }
\usetikzlibrary {datavisualization.formats.functions}

\begin{document}

\pagestyle{fancy}
\fancyhead[L]{Terminale STMG}
\fancyhead[C]{\textbf{TD n°3 : fonctions exponentielles}}
\fancyhead[R]{\today}

La difficulté des exercices est dénotée par des étoiles (de 0 à 3).

\exe{}{
Mettre chacune des expressions ci-dessous sous la forme $a^x$.
\begin{multicols}{2}
\begin{enumerate}[label=(\alph*), itemsep=1ex]
\item $a^5 \times \frac{a^{3,9}}{a^{4,5}}$
\item $\dfrac{a^4}{\dfrac1{a^{1,3}}}$
\item $(a^{2,5} \times \frac{1}{a^-3,8})^{10}$
\item $((a^{0,5})^{2})^3 \times a^{-9,7} \times (a^{5,2})$
\end{enumerate}
\end{multicols}
}{exe:1}{
\begin{multicols}{2}
\begin{enumerate}[label=(\alph*), itemsep=1ex]
\item $a^5 \times \frac{a^{3,9}}{a^{4,5}} = a^{4,4}$
\item $\dfrac{a^4}{\dfrac1{a^{1,3}}} = a^{5,3}$
\item $(a^{2,5} \times \frac{1}{a^-3,8})^{10} = a^{63}$
\item $((a^{0,5})^{2})^3 \times a^{-9,7} \times (a^{5,2}) = a^{-1,5}$
\end{enumerate}
\end{multicols}
}

\exe{}{
Donner le sens de variation des fonctions suivantes 
\begin{multicols}{2}
\begin{enumerate}[label=(\alph*)]
\item $x \mapsto \frac87 \times \left ( 2,02 \right )^x$
\item $x \mapsto \frac14 \times -1 \times \left ( \frac98 \right )^x$
\item $x \mapsto (\frac14 - \frac13) \times 4^x$
\item $x \mapsto -0,1 \times ((2,5)^2 - 6)^x$
\end{enumerate}
\end{multicols}
}{exe:2}{
\begin{multicols}{2}
\begin{enumerate}[label=(\alph*)]
\item $x \mapsto \frac87 \times \left ( 2,02 \right )^x$ est croissante
\item $x \mapsto \frac14 \times -1 \times \left ( \frac98 \right )^x$ est décroissante
\item $x \mapsto (\frac14 - \frac13) \times 4^x$ est décroissante
\item $x \mapsto -0,1 \times ((2,5)^2 - 6)^x$ est croissante
\end{enumerate}
\end{multicols}
}

\exe{, difficulty=1}{
Le PIB de la Corée du Nord est connue depuis 2009 seulement.
En 2009, il était estimé à 28~483,5 Won nord-coréens (abrgégé WNK) et en 2024 à 43~678,9 WNK.
\begin{enumerate}
\item Donner le taux d'évolution globale du PIB nord-coréen sur cette période. 
\item Donner le taux d'évolution \textbf{annuel moyen} du PIB nord-coréen sur cette même période.
\item En utilisant le taux moyen calculé précédemment, donner une estimation :
	\begin{enumerate}
		\item du PIB en 2000 ;
		\item du PIB en 2030.
	\end{enumerate}
\end{enumerate}
}{exe:3}{
\begin{enumerate}
\item $T = \frac{43678,9}{28435,5} - 1 \approx 0,536$.
\item $t = \left ( \frac{43678,9}{28435,5} \right )^{1/15} - 1 \approx 0,029$.
\item 
	\begin{enumerate}
		\item $28435,5 \times (1+t)^{-9} \approx 21~979,30$ WNK.
		\item $43678,9 \times (1+t)^{6} \approx 51~860,4$ WNK.
	\end{enumerate}
\end{enumerate}
}

\exe{, difficulty=1}{ 
Une ancienne élève de STMG décide de créer une entreprise de vente de parfum. Elle estime ses frais d'opération à 300 \euro~ mensuels.
Pour lancer son entreprise, elle contracte un crédit différé à mensualité fixe de 300 \euro. 
Au lancement de son entreprise, elle compte vendre seulement 10 bouteilles à 30 \euro~ le premier mois 
puis faire croître progressivement son affaire pour se stabiliser, après un an, à 80 bouteilles vendues par mois (le prix reste le même).
\begin{enumerate}
\item Calculer le taux de croissance mensuel moyen nécessaire pour tenir les objectifs.
\item Le crédit étant différé, à partir de quel mois l'élève pourra commencer à rembourser son crédit sans risquer de bilan négatif ?
\item Calculer le résultat net de l'entreprise après un an.
\end{enumerate}
\textit{Ne pas oublier de tenir compte des frais d'opération !}
}{exe:4}{
\begin{enumerate}
\item  $t = \left ( \frac{80}{10} \right ) ^ {1/12} - 1 \approx 0,19$.
\item En utilisant le taux moyen, on a : $10 \times (1+t)^4 = 20$, 
donc au 5ème mois l'élève aura un chiffre d'affaire de 600 \euro~ 
ce qui lui permettra de payer ses frais d'opération et son crédit. 
\item Après un an, l'élève vend 80 bouteilles, soit un chiffre d'affaire de 2~400 \euro~, 
ce qui correspond à un résultat net de 1~800 \euro.
\end{enumerate}
}

\newpage
\fancyhead[C]{\textbf{Solutions}}
\shipoutAnswer

\end{document}