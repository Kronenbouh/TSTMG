% Chargement des paquets

\usepackage{amsmath}
\usepackage{amsthm}
\usepackage{amsfonts}
\usepackage{amssymb}
\usepackage{enumerate}
\usepackage{mathtools}
\usepackage{bbm}
\usepackage{xparse, etoolbox}
\usepackage{enumerate}
\usepackage{mathabx}
\usepackage{minted}
\usepackage[french]{babel}
\usepackage{keytheorems}
\usepackage[theorems]{tcolorbox}
\usepackage{hyperref}

% Environnement

%Chargement des paquets
\usepackage{amsmath}
\usepackage{amsthm}
\usepackage{amsfonts}
\usepackage{amssymb}
\usepackage{enumerate}
\usepackage{mathtools}
\usepackage{bbm}
\usepackage{xparse, etoolbox}
\usepackage{enumerate}
\usepackage{mathabx}
\usepackage{minted}
\usepackage[french]{babel}
\usepackage{keytheorems}
\usepackage[theorems]{tcolorbox}
\usepackage{hyperref}
\usepackage[framemethod=tikz]{mdframed}

%Ensembles de nombres
\newcommand{\N} {\mathbb{N}}
\newcommand{\Ne}{\N^\ast}
\newcommand{\Z} {\mathbb{Z}}
\newcommand{\D} {\mathbb{D}}
\newcommand{\Q} {\mathbb{Q}}
\newcommand{\R} {\mathbb{R}}
\newcommand{\Rb}{\overline{\mathbb{R}}}
\newcommand{\Rp}{\R_+}
\newcommand{\Rm}{\R_-}
\newcommand{\K} {\mathbb{K}}
\newcommand{\Cx}{\mathbb{C}}

%Opérateurs
\newcommand{\equi}{\Leftrightarrow}

%Normes
\DeclarePairedDelimiter\abs{\lvert}{\rvert}

%tikz
\usepackage{tikz, pgfplots}
\usetikzlibrary{positioning}
\usetikzlibrary{shapes.geometric}
\usetikzlibrary{positioning}
\usetikzlibrary {angles}
\usepackage{tkz-euclide}

\tikzset{
dot/.style = {circle, fill=#1, minimum size=5pt,
              inner sep=0pt, outer sep=0pt},
dot/.default = black % size of the circle diameter
}

 % for braces
\usetikzlibrary{decorations.pathreplacing}
% for hashing area
\usetikzlibrary{patterns}
% tableaux var, signe
% source https://www.sqlpac.com/fr/documents/latex-package-tkz-tab-tikz-tableaux-de-signes-et-de-variations-de-fonctions.html
\usepackage{tkz-tab}

\tikzset{
	every node/.style = {font=\Large}
}

\tikzset{
	every axis/.style = {clip=true, grid style = {opacity=.5}}
}

%Interface théorème
\renewcommand*{\proofname}{Démonstration}

\theoremstyle{definition}
\newtheorem*{nota}{Notation}

\theoremstyle{definition}
\newtheorem*{conv}{Convention}

\theoremstyle{definition}
\newtheorem{ex}{Exemple}[section]

\theoremstyle{remark}
\newtheorem{rmq}{Remarque}

\theoremstyle{definition}
\newtheorem*{idea}{Idée}

%styles pour théorèmes
\newkeytheoremstyle{tcb-thm}
  {
    headpunct={},
    notebraces={}{},
    noteseparator={ : },
    notefont=\bfseries,
    bodyfont=\slshape,
    tcolorbox={
      arc=0mm,
      colback=blue!5!white,
      colframe=blue!50!black,
      },
  }
  
\newkeytheorem{thm}[
  name=Théorème,
  parent=section,
  style=tcb-thm,
  ]  
  
\newkeytheoremstyle{tcb-prop}
  {
    headpunct={},
    notebraces={}{},
    noteseparator={ : },
    notefont=\bfseries,
    bodyfont=\slshape,
    tcolorbox={
      arc=0mm,
      colback=blue!5!white,
      colframe=blue!75!black,
      },
  }
 
\newkeytheorem{prop}[
  name=Proposition,
  parent=section,
  style=tcb-prop,
  ] 
  
\newkeytheorem{coro}[
  name=Corollaire,
  parent=section,
  style=tcb-prop,
  ]
  
\newkeytheoremstyle{tcb-lem}
  {
    headpunct={},
    notebraces={}{},
    noteseparator={ : },
    notefont=\bfseries,
    bodyfont=\slshape,
    tcolorbox={
      arc=0mm,
      colback=blue!5!white,
      colframe=blue!100!black,
      },
  }
  
\newkeytheorem{lem}[
  name=Lemme,
  parent=section,
  style=tcb-lem,
  ]
 
  
\newkeytheoremstyle{tcb-def}
  {
    headpunct={},
    notebraces={}{},
    noteseparator={ : },
    notefont=\bfseries,
    bodyfont=\slshape,
    tcolorbox={
      arc=0mm,
      colback=orange!5!white,
      colframe=orange!75!black,
      },
  }
  
\newkeytheorem{defn}[
  name=Définition,
  parent=section,
  style=tcb-def,
  ]
 
\newkeytheorem{rapl}[
  name=Rappel,
  style=tcb-def,
  ]
  
\newkeytheoremstyle{tcb-exo}
  {
    headpunct={},
    notebraces={}{},
    noteseparator={ : },
    notefont=\bfseries,
    bodyfont=\slshape,
    tcolorbox={
      arc=0mm,
      colframe=black,
      colback=white,
      },
  }
  
\newkeytheorem{exo}[
  name=Exercice,
  style=tcb-exo,
  ]
  
\surroundwithmdframed[
	hidealllines=true,
	leftline=true,
	innerleftmargin=10pt,
	innerrightmargin=10pt,
	innertopmargin=-4pt,
	nobreak=true,
]{proof}

% Commandes perso

%Ensembles de nombres
\newcommand{\N} {\mathbb{N}}
\newcommand{\Ne}{\N^\ast}
\newcommand{\Z} {\mathbb{Z}}
\newcommand{\D} {\mathbb{D}}
\newcommand{\Q} {\mathbb{Q}}
\newcommand{\R} {\mathbb{R}}
\newcommand{\Rb}{\overline{\mathbb{R}}}
\newcommand{\Rp}{\R_+}
\newcommand{\Rm}{\R_-}
\newcommand{\K} {\mathbb{K}}
\newcommand{\Cx}{\mathbb{C}}

%Opérateurs
\newcommand{\equi}{\Leftrightarrow}

%Normes
\DeclarePairedDelimiter\abs{\lvert}{\rvert}

%Commande d'exo
\newcommand{\exe}[4]{
	\begin{Exercise}[title=#1, label=#3]
		\marginpar{\mbox{\scriptsize(solution p.\pageref{\ExerciseLabel-Answer})}}
		#2
	\end{Exercise}
	\begin{Answer}[ref=#3]
		#4
	\end{Answer}
}

%tikz
\usepackage{tikz, pgfplots}
\usetikzlibrary{positioning}
\usetikzlibrary{shapes.geometric}
\usetikzlibrary{positioning}
\usetikzlibrary {angles}
\usepackage{tkz-euclide}

\tikzset{
dot/.style = {circle, fill=#1, minimum size=5pt,
              inner sep=0pt, outer sep=0pt},
dot/.default = black % size of the circle diameter
}

 % for braces
\usetikzlibrary{decorations.pathreplacing}
% for hashing area
\usetikzlibrary{patterns}
% tableaux var, signe
% source https://www.sqlpac.com/fr/documents/latex-package-tkz-tab-tikz-tableaux-de-signes-et-de-variations-de-fonctions.html
\usepackage{tkz-tab}

\tikzset{
	every node/.style = {font=\Large}
}

\tikzset{
	every axis/.style = {clip=true, grid style = {opacity=.5}}
}
\usepackage{graphicx,wrapfig}
\usepackage{eurosym}
\usepackage{varwidth}
\graphicspath{ {./images/} }
\usetikzlibrary {datavisualization.formats.functions}

\begin{document}

\pagestyle{fancy}
\fancyhead[L]{Terminale STMG}
\fancyhead[C]{\textbf{TD n°4 : fonction inverse}}
\fancyhead[R]{\today}

La difficulté des exercices est dénotée par des étoiles (de 0 à 3).

\exe{}{
Donner les fonctions dérivées des fonctions suivantes sur $\Ret$ :
\begin{multicols}{2}
\begin{enumerate}
\item $x \mapsto \frac7{x} + 3x$
\item $x \mapsto 4x^2 - 2x - \frac3{x}$
\item $x \mapsto \frac32 \times \frac1{x} + x$
\item$x \mapsto 8x^3 - 7x^2 + 6x -2 - \frac{12}{x}$
\end{enumerate}
\end{multicols}
}{exe:1}{

}

\exe{}{
Etudier les variations des fonctions suivantes sur $\R$ :
\begin{multicols}{2}
\begin{enumerate}
\item $x \mapsto 2x^2 - 8x + 25$
\item $x \mapsto 4x^2 - 2x + \frac73$
\item $x \mapsto \frac32 x^2 + 9x - 5$
\item $x \mapsto \frac17x^2 + \frac23 x - \pi$
\end{enumerate}
\end{multicols}
}{exe:2}{

}

\exe{}{
Etudier les variations des fonctions suivantes sur $\Ret$ :
\begin{multicols}{2}
\begin{enumerate}
\item $2x-12 + \frac{162}{x}$
\item $7x - 426 - \frac{2}{x}$
\item $-10x + 20 - \frac{360}{x}$
\item $-7x + 312 + \frac{12}{x}$
\end{enumerate}
\end{multicols}
\textit{Indication : on pourra utiliser le fait que, pour $a, b, c \in \R, c \neq 0, a + \frac{b}{c} = \frac{ac + b}{c}$.}
}{exe:3}{

}

\exe{}{
On considère la fonction $f$ définie sur $\R$ par $f(x) = x^3 - \frac92 x^2 +2x +15$.
\begin{enumerate}
\item Déterminer la fonction dérivée de $f$ sur $\R$.
\item Montrer que l'on peut écrire :
\[ f'(x) = 3(x-2)(x-1) \]
\item En déduire les valeurs $x$ pour lesquels $f'(x)=0$.
\item Dresser le tableau de variation de $f$ sur $\R$.
\end{enumerate}
\textit{On rappelle que $a^2-b^2=(a-b)(a+b)$}.
}{exe:4}{

}

\exe{, difficulty=1}{
On considère la fonction $g$ définie sur $\Ret$ par $g(x) = 5x - 3 + \frac{20}{x}$.
\begin{enumerate}
\item Déterminer la fonction dérivée de $g$ sur $\Ret$.
\item Montrer que l'on peut écrire :
\[ g'(x) = \dfrac{5(x-2)(x+2)}{x^2} \]
\item En déduire les valeurs $x$ pour lesquels $g'(x)=0$.
\item Etudier les limites éventuelles de $g$ sur son ensemble de définition.
\item Dresser le tableau de variation de $g$ sur $\Ret$.
\end{enumerate}
}{exe:5}{

}

\exe{, difficulty=1}{
On considère la fonction $g$ définie sur $\Ret$ par $g(x) = 3x - 62 + \frac{48}{x}$
\begin{enumerate}
\item Déterminer la fonction dérivée de $g$ sur $\Ret$.
\item Montrer que l'on peut écrire :
\[ g'(x) = \dfrac{3(x-4)(x+4)}{x^2} \]
\item Etudier les limites éventuelles de $g$ sur son ensemble de définition.
\item Dresser le tableau de variation de $g$ sur $\Ret$.
\end{enumerate}
}{exe:6}{

}

\exe{, difficulty=1}{
Une entreprise produit des automates industriels. Elle décide d'estimer leur valeur à la vente selon la méthode suivante :
\begin{enumerate}[label=(\roman*)]
\item L'entreprise décompte les erreurs commises par les automates sur 1000 tâches. Elle note $x$ ce nombre d'erreurs.
\item La valeur du robot est ensuite fixée en calculant le quotient $\frac{100~000}{x}$.
\end{enumerate}
\begin{enumerate}
\item La valeur du robot décroît-elle en fonction de nombres d'erreurs commises ? Justifier.
\item Existe-t-il un cas pour lequel cette méthode est problématique ? Justifier.
\end{enumerate}
}{exe:7}{

}

\exe{, difficulty=2}{
Une enreprise désire commercialiser un produit.
Elle se fixe un objectif de résultat net de 200 000 \euro.~
Le coût de production d'un produit est de 5 \euro~ par unité.
Les coûts de production fixes sont de 48 000 \euro.
\begin{enumerate}
\item Démontrer que le coût de production de $x$ objets est donné par la formule $c(x) = 78~000 + 5x$.
\item Démontrer que le prix de vente unitaire $p(x)$ minimal pour atteindre l'objectif de résultat est donné par :
\[ p(x) = \frac{122~000}{x} + 5 . \]
Ce prix dépend-il du nombre de produits que l'entreprise pense pouvoir vendre ?
\item On suppose que l'enteprise vend 20 000 produits. Quel est le prix de vente unitaire minimal pour atteindre l'objectif de résultat ?
\item On suppose que l'entreprise dispose d'un marché illimité (elle peut vendre autant de produits qu'elle le souhaite).
Quel est le prix de vente unitaire minimal pour atteindre l'objectif de résultat ?
\end{enumerate}
}{exe:8}{

}

\newpage
\fancyhead[C]{\textbf{Solutions}}
\shipoutAnswer

\end{document}