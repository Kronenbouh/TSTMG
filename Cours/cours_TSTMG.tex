\documentclass[a4paper, 12pt]{report}

% Chargement des paquets

\usepackage{amsmath}
\usepackage{amsthm}
\usepackage{amsfonts}
\usepackage{amssymb}
\usepackage{enumerate}
\usepackage{mathtools}
\usepackage{bbm}
\usepackage{xparse, etoolbox}
\usepackage{enumerate}
\usepackage{mathabx}
\usepackage{minted}
\usepackage[french]{babel}
\usepackage{keytheorems}
\usepackage[theorems]{tcolorbox}
\usepackage{hyperref}

% Environnement

%Chargement des paquets
\usepackage{amsmath}
\usepackage{amsthm}
\usepackage{amsfonts}
\usepackage{amssymb}
\usepackage{enumerate}
\usepackage{mathtools}
\usepackage{bbm}
\usepackage{xparse, etoolbox}
\usepackage{enumerate}
\usepackage{mathabx}
\usepackage{minted}
\usepackage[french]{babel}
\usepackage{keytheorems}
\usepackage[theorems]{tcolorbox}
\usepackage{hyperref}
\usepackage[framemethod=tikz]{mdframed}

%Ensembles de nombres
\newcommand{\N} {\mathbb{N}}
\newcommand{\Ne}{\N^\ast}
\newcommand{\Z} {\mathbb{Z}}
\newcommand{\D} {\mathbb{D}}
\newcommand{\Q} {\mathbb{Q}}
\newcommand{\R} {\mathbb{R}}
\newcommand{\Rb}{\overline{\mathbb{R}}}
\newcommand{\Rp}{\R_+}
\newcommand{\Rm}{\R_-}
\newcommand{\K} {\mathbb{K}}
\newcommand{\Cx}{\mathbb{C}}

%Opérateurs
\newcommand{\equi}{\Leftrightarrow}

%Normes
\DeclarePairedDelimiter\abs{\lvert}{\rvert}

%tikz
\usepackage{tikz, pgfplots}
\usetikzlibrary{positioning}
\usetikzlibrary{shapes.geometric}
\usetikzlibrary{positioning}
\usetikzlibrary {angles}
\usepackage{tkz-euclide}

\tikzset{
dot/.style = {circle, fill=#1, minimum size=5pt,
              inner sep=0pt, outer sep=0pt},
dot/.default = black % size of the circle diameter
}

 % for braces
\usetikzlibrary{decorations.pathreplacing}
% for hashing area
\usetikzlibrary{patterns}
% tableaux var, signe
% source https://www.sqlpac.com/fr/documents/latex-package-tkz-tab-tikz-tableaux-de-signes-et-de-variations-de-fonctions.html
\usepackage{tkz-tab}

\tikzset{
	every node/.style = {font=\Large}
}

\tikzset{
	every axis/.style = {clip=true, grid style = {opacity=.5}}
}

%Interface théorème
\renewcommand*{\proofname}{Démonstration}

\theoremstyle{definition}
\newtheorem*{nota}{Notation}

\theoremstyle{definition}
\newtheorem*{conv}{Convention}

\theoremstyle{definition}
\newtheorem{ex}{Exemple}[section]

\theoremstyle{remark}
\newtheorem{rmq}{Remarque}

\theoremstyle{definition}
\newtheorem*{idea}{Idée}

%styles pour théorèmes
\newkeytheoremstyle{tcb-thm}
  {
    headpunct={},
    notebraces={}{},
    noteseparator={ : },
    notefont=\bfseries,
    bodyfont=\slshape,
    tcolorbox={
      arc=0mm,
      colback=blue!5!white,
      colframe=blue!50!black,
      },
  }
  
\newkeytheorem{thm}[
  name=Théorème,
  parent=section,
  style=tcb-thm,
  ]  
  
\newkeytheoremstyle{tcb-prop}
  {
    headpunct={},
    notebraces={}{},
    noteseparator={ : },
    notefont=\bfseries,
    bodyfont=\slshape,
    tcolorbox={
      arc=0mm,
      colback=blue!5!white,
      colframe=blue!75!black,
      },
  }
 
\newkeytheorem{prop}[
  name=Proposition,
  parent=section,
  style=tcb-prop,
  ] 
  
\newkeytheorem{coro}[
  name=Corollaire,
  parent=section,
  style=tcb-prop,
  ]
  
\newkeytheoremstyle{tcb-lem}
  {
    headpunct={},
    notebraces={}{},
    noteseparator={ : },
    notefont=\bfseries,
    bodyfont=\slshape,
    tcolorbox={
      arc=0mm,
      colback=blue!5!white,
      colframe=blue!100!black,
      },
  }
  
\newkeytheorem{lem}[
  name=Lemme,
  parent=section,
  style=tcb-lem,
  ]
 
  
\newkeytheoremstyle{tcb-def}
  {
    headpunct={},
    notebraces={}{},
    noteseparator={ : },
    notefont=\bfseries,
    bodyfont=\slshape,
    tcolorbox={
      arc=0mm,
      colback=orange!5!white,
      colframe=orange!75!black,
      },
  }
  
\newkeytheorem{defn}[
  name=Définition,
  parent=section,
  style=tcb-def,
  ]
 
\newkeytheorem{rapl}[
  name=Rappel,
  style=tcb-def,
  ]
  
\newkeytheoremstyle{tcb-exo}
  {
    headpunct={},
    notebraces={}{},
    noteseparator={ : },
    notefont=\bfseries,
    bodyfont=\slshape,
    tcolorbox={
      arc=0mm,
      colframe=black,
      colback=white,
      },
  }
  
\newkeytheorem{exo}[
  name=Exercice,
  style=tcb-exo,
  ]
  
\surroundwithmdframed[
	hidealllines=true,
	leftline=true,
	innerleftmargin=10pt,
	innerrightmargin=10pt,
	innertopmargin=-4pt,
	nobreak=true,
]{proof}

% Commandes perso

%Ensembles de nombres
\newcommand{\N} {\mathbb{N}}
\newcommand{\Ne}{\N^\ast}
\newcommand{\Z} {\mathbb{Z}}
\newcommand{\D} {\mathbb{D}}
\newcommand{\Q} {\mathbb{Q}}
\newcommand{\R} {\mathbb{R}}
\newcommand{\Rb}{\overline{\mathbb{R}}}
\newcommand{\Rp}{\R_+}
\newcommand{\Rm}{\R_-}
\newcommand{\K} {\mathbb{K}}
\newcommand{\Cx}{\mathbb{C}}

%Opérateurs
\newcommand{\equi}{\Leftrightarrow}

%Normes
\DeclarePairedDelimiter\abs{\lvert}{\rvert}

%Commande d'exo
\newcommand{\exe}[4]{
	\begin{Exercise}[title=#1, label=#3]
		\marginpar{\mbox{\scriptsize(solution p.\pageref{\ExerciseLabel-Answer})}}
		#2
	\end{Exercise}
	\begin{Answer}[ref=#3]
		#4
	\end{Answer}
}

%tikz
\usepackage{tikz, pgfplots}
\usetikzlibrary{positioning}
\usetikzlibrary{shapes.geometric}
\usetikzlibrary{positioning}
\usetikzlibrary {angles}
\usepackage{tkz-euclide}

\tikzset{
dot/.style = {circle, fill=#1, minimum size=5pt,
              inner sep=0pt, outer sep=0pt},
dot/.default = black % size of the circle diameter
}

 % for braces
\usetikzlibrary{decorations.pathreplacing}
% for hashing area
\usetikzlibrary{patterns}
% tableaux var, signe
% source https://www.sqlpac.com/fr/documents/latex-package-tkz-tab-tikz-tableaux-de-signes-et-de-variations-de-fonctions.html
\usepackage{tkz-tab}

\tikzset{
	every node/.style = {font=\Large}
}

\tikzset{
	every axis/.style = {clip=true, grid style = {opacity=.5}}
}

\begin{document}

\chapter{Suites numériques}

\section{Définition générale}

Considérons un ensemble $E$ dont on ne précise pas la nature et qui contient un nombre éventuellement infini de points. Si on imagine que cet ensemble est une urne dans laquelle on tire successivement, avec remise, $N$ points, on peut alors construire une famille de points :
\[ \{x_1, x_2, \dots, x_N \}, \]
où le point $x_1$ est le premier point tiré, $x_2$ le deuxième, etc., j'usqu'au point $x_N$ (avec $N \geq 1$, éventuellement infini). On dit alors que les points sont \textbf{indexés} par les nombres entiers. \\
Comme le tirage a été effectué avec remise, il est possible que, pour deux rangs différents $i \neq j$, on ait $x_i = x_j$, c'est-à-dire que le même point soit indexé par deux entiers distincts. \\

\begin{center}

 \begin{tikzpicture}[scale=1.2]
 
 % domaine
\coordinate (B) at (0,1);
\coordinate (C) at (.1,.9);
	
\node[dot=red, label=above:$3$] at (C){};
	    	
\node[ellipse, draw, label=above:$\N$, minimum height=110pt, minimum width=80pt] at (B){};
	
% codomaine
\coordinate (D) at (4,.7);
\node[dot=red, label=below:$x_3$] at (D){};
\node[ellipse, draw, label=above:$E$, minimum height=110pt, minimum width=80pt] at (4,1){};
	
%maps	
\draw[thick, ->] (C) -- (3.9,.72);
\draw[thick, ->] (.6,1.5) -- (4,1.32);
\draw[thick, ->] (.2,-.1) -- (4,1.18);
	    	
% morepoints  	
%left ones
\node[dot=black] at (0, 2){};		    	
\node[dot=black] at (.2,-.1){};
\node[dot=black] at (.6,1.5){};
\node[dot=black] at (.75,.75){};
\node[dot=black] at (-.3,1.4){};
\node[dot=black] at (-.5,.2){};
	    
%rightones
\node[dot=black] at (4,2){};
\node[dot=black] at (4.65,.75){};
\node[dot=black] at (4.1,1.25){};
\node[dot=black] at (4.6,1.35){};
\node[dot=black]  at (3.45,.45){};
\node[dot=black]  at (3.3,1.15){};
\node[dot=black]  at (4,-0.2){};

\end{tikzpicture}

\textit{Une illustration de l'indexation des éléments de E}

\end{center}

Si, pour l'ensemble $E$, on choisit l'ensemble des nombres réels $\R$ et que l'on applique ce même processus d'indexation, alors on aura définit une suite numérique $(x_n)$ dont l'étude est tout l'objet de ce chapitre.

\begin{defn}
On appelle \textbf{suite numérique} toute application de $D \subset \N$ dans $\R$, c'est à dire une application de la forme :
\[ \begin{array}{c c c c}
u : & \N & \to & \R \\
& n & \mapsto & u_n
\end{array} \]
On notera cette suite $(u_n)_{n \in \N}$ ou parfois simplement $(u_n)$.
\end{defn}

\begin{nota}
On fera attention à ne pas confondre l'objet $(u_n)_{n \in \N}$ qui est une application de $\N$ dans $\R$, (c'est-à-dire un objet qui ``transforme'' un index en un nombre réel), avec l'objet $u_n$, qui désigne le nombre réel associé à l'index $n$. \\
On écrira donc $(u_n)_{n \in \N}$ ou simplement $(u_n)$ (toujours avec des parenthèses) pour désigner \textbf{la suite numérique} et on écrira $u_n$ pour désigner \textbf{le terme de rang $n$}.
\end{nota} 

\begin{rmq}
On pourra rencontrer des suites dont le domaine de définition n'est pas $\N$ tout entier mais un sous ensemble $D \subset \N$ (par exemple les entiers non nuls). Ce petit détail ne change néanmoins pas notre propos.
Dans la suite de ce cours, on considèrera donc des suites dont le premier terme est $u_0$.
\end{rmq}

\section{Expression d'une suite}

On sait maintenant ce qu'est une suite, mais cette connaissance est entièrement abstraite, \textbf{c'est une application sur $\N$, d'accord, mais que fait-elle exactement ?} Pour répondre à cette question, on doit s'intéresser aux modes de définition d'une suite.

\begin{defn}[Définition explicite]
On dit qu'une suite numérique $(u_n)_{n \in \N}$ est définie de façon explicite lorsque l'on dispose d'une expression fonctionnelle permettant de calculer, pour tout index $n$, le réel $u_n$. En notant $f$ cette expression, on a alors 
\[ u_n = f(n) \text{ pour tout } n \in \N \]
\end{defn}

\begin{ex}
La suite définie sur $\N$ par $u_n = 2n +1$ est définie de façon explicite. On remarquera que $u_0 = 1, u_1 = 3, u_2 = 5$ etc. Cette suite est en fait la suite des nombres impairs.
\end{ex}

\begin{defn}[Définition par récurrence]
On dit qu'une suite numérique $(u_n)_{n \in \N}$ est définie par récurrence lorsque sont donnés son premier terme $u_0$ et une relation permettant de calculer le terme de rang $n+1$ à partir du terme de rang $n$. On a alors
\[ u_0 = x \in \R \qquad u_{n+1} = f(u_n) \]
\end{defn}

\begin{ex}
La suite définie sur $\N$ par $u_0=1$ et $u_{n+1} = u_n + 2$ est définie par récurrence. 
On remarquera que $u_1 = 3, u_2 = 5$ etc. C'est à nouveau la suite des nombres impairs !
\end{ex}

\section{Suite arithmétique}

\begin{defn}
On appelle suite arithmétique de raison $r \in \R$ et de premier terme $u_0 \in \R$, une suite définie par la relation de récurrence 
\[ u_{n+1} = u_n + r .\]
\end{defn}

\begin{prop}
Soit $(u_n)$ une suite arithmétique. Lorsqu'on représente les points $(n, u_n)$ pour $n \in \N$ dans un repère orthonormé, ils sont alignés.
\end{prop}

\begin{ex}
On considère la suite arithmétique de premier terme $u_0=1$ et de raison $r=2$. La représentation graphique des points $(n, u_n)$ pour $n \in \{ 0, 1, 2, 3 \}$ est donnée ci-dessous :
\begin{center}
\begin{tikzpicture}[>=stealth, scale=1]
	\begin{axis}[xmin = -1, xmax= 3.5, ymin=-1, ymax=8, axis x line=middle, axis y line=middle, axis line style=<->,
	xlabel={}, ylabel={}, grid=both, grid style = {opacity=.5}]]
	\addplot [only marks] table {
	0 1
	1 3
	2 5
	3 7
	};
	\addplot[black, thin, domain =0:3, samples=2] {2*x+1};
	\end{axis}
\end{tikzpicture}
\end{center}
\end{ex}

\begin{thm}
On considère une suite arithmétique $(u_n)$ de premier terme $u_0$ et de raison $r$. Alors, l'expression 
\[ u_n = u_0 + n \cdot r \]
est une définition explicite de la suite $(u_n)$.
\end{thm}

\begin{proof}
On remarque que :
\begin{align*}
u_1 & = u_0 + r \\
u_2 & = u_1 + r = u_0 + 2r  \\
u_3 & = u_2 + r = u_0 + 2r + r = u_0 + 3r 
\end{align*}
Donc, on procédant comme ceci, de proche en proche, on déduit que 
\[ u_n = u_0 + n \cdot r . \]
\end{proof}

\begin{rmq} On peut donc calculer n'importe quel terme d'une suite arithmétique sans connaissance explicite des termes de rang inférieur (mis à part le premier). Cette proposition parait intuitive lorsqu'on observe la représentation graphique de quelques termes successifs d'une suite arithmétique, comme ils sont alignés, il suffit de connaître deux paramètres (la pente de la droite et sa position pour une abscisse donnée) pour connaître toute la droite. 
\end{rmq}

\begin{nota}
On utilise le symbole $\sum$ pour désigner une somme de plusieurs éléments indexés. Par exemple, les termes d'une suite numérique $(u_n)$ sont des nombres réels indexés par $n$. On écrit alors :
\[ u_0 + u_1 + u_2 + \dots + u_N = \sum_{n=0}^N u_n \]
pour désigner la somme des termes de rang 0 à $N$ de la suite $(u_n)$.
\end{nota}

\begin{lem}
La somme des $n$ premiers entiers (en partant de 1) est égale à 
\[ 1+2+ \dots +n = \dfrac{n(n+1)}2 \]
\end{lem}

\begin{proof}
Essayons d'abord de calculer $1+2+3+\dots+98+99+100$. On remarque que $1+100=101$, $2+99=101$, $3+98=101 \dots$ Donc en regroupant les entiers deux par deux, on pourra former 50 $(=100 \div 2)$ couples dont la somme sera 101. On va donc trouver :
\[  1+2+3+\dots+98+99+100 = \dfrac{100(100+1)}2  \]
Ce qui est le résultat annoncé. \\

Revenons maintenant au cas général, si $n$ est pair on va former $\dfrac{n}2$ couples de deux entiers dont la somme sera $n+1$ et on trouvera bien 
\[ 1+2+ \dots +n = \dfrac{n(n+1)}2. \]
Si $n$ est impair, on va former cette fois $\dfrac{n-1}2$ couples de deux entiers dont la somme sera $n+1$ et on aura un entier isolé dont la valeur sera ``pile au milieu'', c'est à dire $\dfrac{n+1}2$. 
Finalement, on aura 
\[ 1+2+ \dots +n = \dfrac{(n-1)(n+1)}2 + \dfrac{n+1}2 = \dfrac{n(n+1)}2. \]
\end{proof}

\begin{prop}
Soit $(u_n)$ une suite arithmétique de raison $r$ et de premier terme $u_0$ et soit $N \in \N$. La somme des $N+1$ premiers termes de la suite $(u_n)_{n \in \N}$ est donnée par la formule 
\[ \sum_{n=0}^N u_n = u_0+ u_1 + \dots u_N =  \dfrac{N+1}{2} \cdot (u_0 + u_N). \]
\end{prop}

\begin{rmq}
Attention, on somme bien $N+1$ termes (et non pas $N$) car le premier terme est indexé par 0 !
\end{rmq}

\begin{proof}
%%% Preuve alternative
%On va utiliser la formule $u_n = u_0 + nr$. En substituant dans $u_0+ u_1 + \dots u_N$, on trouve :
%\begin{align*}
%u_0+ u_1 + u_2 + \dots u_N & = u_0 + u_0 + r + u_0 + 2r + \dots + u_0 + Nr \\
%& = (N+1) u_0 + r + 2r + \dots + Nr \\
%& = (N+1) u_0 + r \times (1+2+\dots+N) \\
%& = (N+1) u_0 + r \times \dfrac{N(N+1)}2 \\
%& = (N+1) ( u_0 + r \times \dfrac{N}2 \\
%& = (N+1) \cdot \dfrac{2u_0 + N r}2 \\
%& = (N+1) \cdot \dfrac{u_0 + u_0 + Nr}2 \\
%& = (N+1) \cdot \dfrac{u_0 + u_N}2 \qquad \text{car } u_N = u_0 + Nr \\
%& =  \dfrac{N+1}{2} \cdot (u_0 + u_N) 
%\end{align*}
On va utiliser la même méthode que pour la somme $1+2+\dots+n$. Commençons par un exemple simple, la suite arithmétique de premier terme $u_0=1$ et de raison $r=2$. On veut calculer $u_0 + u_1 + \dots + u_{100}$.
On remarque que :
\begin{align*} 
u_0 = 0, ~ & u_{100} = 1 + 100 \times 2 = 201 & \text{ et } & u_0 + u_{100} = 202 \\
u_1 = 3, ~ & u_{99} = 1 + 99 \times 2 = 199 & \text{ et } & u_1 + u_{99} = 202 \\
\text{etc} \dots & & &
\end{align*}

Comme on additionne 101 termes, on va pouvoir former 50 couples de deux termes dont la somme vaudra $202=u_0+u_100$. Seul le terme 
\[ u_{50}=1+50 \times 2 = 101 = \dfrac{u_0+u_{100}}{2} \] 
sera isolé. Au final
\[ u_0 + u_1 + \dots + u_{100} = \dfrac{100}{2} (u_0+u_{100}) \dfrac{u_0+u_100}{2} = \dfrac{101}{2} (u_0+u_{100}) \]

Revenons maintenant au cas général, si $N$ est pair on va former $\frac{N}2$ couples de deux termes dont la somme sera $u_0+u_N$ et il restera le terme $u_{N/2}$ isolé. Or
\[ u_{N/2} = u_0 + \dfrac{N}{2} r = \dfrac{u_0 + u_0 + Nr}{2} = \dfrac{u_0 + u_N}{2}, \]
car $u_N = u_0 + Nr$. Et finalement
\[ u_0+ u_1 + \dots u_N = \dfrac{N+1}2 (u_0 + u_N) \] \\

Si $N$ est impair, on va former $\frac{N+1}2$ couples de deux termes dont la somme sera $u_0+u_N$ et aucun terme ne sera isolé. On aura donc directement la formule cherchée.
\end{proof}

\section{Suite géométrique}

\begin{defn}
On appelle suite géométrique de raison $q \in \R$ et de premier terme $u_0 \in \R$, une suite définie par la relation de récurrence 
\[ u_{n+1} = u_n \cdot  q.\]
\end{defn}

\begin{prop}
Soit $(u_n)$ une suite géométrique. Lorsqu'on représente les points $(n, u_n)$ pour $n \in \N$ dans un repère orthonormé, \textbf{ils ne sont pas alignés.}
\end{prop}

\begin{ex}
On considère la suite géométrique de premier terme $u_0=1$ et de raison $q=2$. La représentation graphique des points $(n, u_n)$ pour $n \in \{ 0, 1, 2, 3 \}$ est donnée ci-dessous :
\begin{center}
\begin{tikzpicture}[>=stealth, scale=1]
	\begin{axis}[xmin = -1, xmax= 3.5, ymin=-1, ymax=8, axis x line=middle, axis y line=middle, axis line style=<->,
	xlabel={}, ylabel={}, grid=both, grid style = {opacity=.5}]]
	\addplot [only marks] table {
	0 1
	1 2
	2 4
	3 8
	};
	\addplot[black, thin, domain =0:3] {pow(2,x)};
	\end{axis}
\end{tikzpicture}
\end{center}
\textit{Ici les termes de la suite croissent de façon exponentielle. Ceci est dû au fait que la raison de la suite est strictement supérieur à 1.}
\end{ex}

\begin{ex}
On considère la suite géométrique de premier terme $u_0=1$ et de raison $q=\frac12$. La représentation graphique des points $(n, u_n)$ pour $n \in \{ 0, 1, 2, 3 \}$ est donnée ci-dessous :
\begin{center}
\begin{tikzpicture}[>=stealth, scale=1]
	\begin{axis}[xmin = -1, xmax= 3.5, ymin=-0.25, ymax=1.25, axis x line=middle, axis y line=middle, axis line style=<->,
	xlabel={}, ylabel={}, grid=both, grid style = {opacity=.5}]]
	\addplot [only marks] table {
	0 1
	1 0.5
	2 0.25
	3 0.125
	};
	\addplot[black, thin, domain =0:3] {pow(0.5,x)};
	\end{axis}
\end{tikzpicture}
\end{center}
\textit{Ici les termes de la suite décroissent de façon exponentielle. Ceci est dû au fait que la raison de la suite est strictement inférieur à 1 (mais positive).}
\end{ex}

\begin{thm}
On considère une suite géomérique $(u_n)$ de premier terme $u_0$ et de raison $q$. Alors, l'expression 
\[ u_n = u_0 \cdot q^n \]
est une définition explicite de la suite $(u_n)$.
\end{thm}

\begin{proof}
On remarque que :
\begin{align*}
u_1 & = u_0 \cdot q \\
u_2 & = u_1 \cdot q = u_0 \cdot q \cdot q = u_0 \cdot q^2  \\
u_3 & = u_2 \cdot q = u_0 \cdot q \cdot q \cdot q = u_0 \cdot q^3
\end{align*}
Donc, on procédant comme ceci, de proche en proche, on déduit que 
\[ u_n = u_0 \cdot q^n . \]
\end{proof}

\begin{rmq}
Cette proposition est l'équivalent géométrique de la proposition déjà formulée pour les suites arithmétiques
\end{rmq}

\begin{prop}
Soit $(u_n)$ une suite géométrique de raison $q$ et de premier terme $u_0$ et soit $N \in \N$. La somme des $N+1$ premiers termes de la suite $(u_n)_{n \in \N}$ est donnée par la formule 
\[ \sum_{n=0}^N u_n = u_0 + u_1 + \dots + u_N = u_0 \cdot \frac{1-q^{N+1}}{1-q}. \]
\end{prop}

\begin{proof}
Notons $S = u_0 + u_1 + \dots + u_N$. On remarque que 
\[ qS = q u_0 + q u_1 + \dots + q u_N . \] 
Or, pour tout $n \in \N$, on a $q \cdot u_n = u_{n+1}$ selon la définition d'une suite géométrique. Donc
\begin{align*} 
qS & = u_1 + u_2 + \dots u_{N+1} \\
& = S - u_{N+1} + u_0 \\
& = S - u_0 \cdot q^{N+1} + u_0 
\end{align*}
et donc
\[ S - qS = u_0 (1-q^{N+1}) \equi S = u_0 \dfrac{1-q^{N+1}}{1-q} .\]
\end{proof}

\chapter{Fonctions logarithmes}

\section{Définition}

\begin{defn}[Logarithme décimal]
Soit $a$ un réel strictement positif. On appelle \textbf{logarithme décimal} de $a$ l'unique solution, dans $\R$, de l'équation 
\[ 10^x = a . \]
On note cette solution $\log(a)$.
\end{defn}

\begin{rmq}
Autrement dit, $10^{\log(a)} = a$. De plus, la fonction $\log$ est \textbf{définie sur $]0, \infty[$ uniquement.}
\end{rmq}

\begin{ex}
Le nombre réel $\log(2)$ est \textbf{le seul} nombre réel vérifiant $10^x = 2$.
\end{ex}

\begin{defn}[Fonction logarithme décimal]
On définit la fonction logarithme décimal sur l'intervalle $]0, \infty[$ par :
\[ \begin{array}{c c c c} \log : & ]0, \infty[ & \to & \R \\ & x & \mapsto & \log(x) \end{array} \]
La fonction lograithme décimal est donc la fonction qui, a un réel strictement positif, associe le logarithme décimal de ce réel.
\end{defn}

\begin{rmq}
Dans la suite de ce cours, on étudiera \textbf{la fonction} logarithme décimal plutôt que \textbf{la valeur} prise ponctuellement par cette fonction en un réel $a >0$ donné.
\end{rmq}

\begin{rmq}
On remarque que $10^x = 1 \equi x = 0$ donc $\log(1) = 0$.
\end{rmq}

\begin{prop}[Continuité]
La fonction $\log$ est continue sur $]0, \infty[$.
\end{prop}

\begin{rmq}
Une conséquence de ce théorème est que la courbe représentative de la fonction $\log$ dans un repère orthonormée est ``lisse'', elle ne présente aucune discontinuité. Cette représentation est donnée ci-dessous :
	\begin{center}
	\begin{tikzpicture}[>=stealth, scale=1]
		\coordinate (OR) at (0.00, 0.00);
		\coordinate (LX) at (-1.00, 0.00); % left x
		\coordinate (RX) at (8.00, 0.00); % right x
		\coordinate (BY) at (0.00, -5.00); % bottom y
		\coordinate (TY) at (0.00, 3.00); % top y
		%
		% axa 0x
		%
		\draw[->][line width=1.00pt] (LX) -- (RX);
		\node[blue] at (7.8,-0.4) {\textbf{\textit{x}}};
		%
		% axa 0y
		%
		\draw[->][line width=1.00pt] (BY) -- (TY);
		\filldraw [red] (OR) circle(2pt);
		\node[red] at (0.2,-0.3) {\textbf{\textit{0}}};
		\node[right,blue] at (0.2, 2.8) {\textbf{\textit{ln x}}};
		\draw[blue, line width=1.75pt, domain=0.005:7.00,samples=500] plot[smooth](\x, {ln(\x)});
	\end{tikzpicture}
	\end{center}
\end{rmq}

\section{Propriétés}

\begin{prop}[Sens de variation]
La fonction $\log$ est strictement croissante sur $]0, \infty[$.
\end{prop}

\begin{proof}
Soient $a, b \in ]0, \infty[$ tels que $a< b$. Comme la fonction $x \mapsto 10^x$ est strictement croissante sur $\R$, 
on a bien $\log(a) < \log(b)$.
\end{proof}

\begin{coro}
Pour tous réels $a, b$ strictement positifs
\begin{enumerate}[(i)]
\item $\log(a) = \log(b)$ équivaut à $a=b$.
\item $a < b$ équivaut à $\log(a) < \log(b)$. 
\item En particulier $0 < a <1 \equi \log(a) < 0$ et $a>1\equi \log(a)>0$.
\end{enumerate}
\end{coro}

\begin{prop}[Seuil]
Soit $K$ un réel strictement positif, on a :
\[ \log(a) < K \equi a < 10^K. \]
\end{prop}

\begin{prop}[Propriétés algébriques]
Soient $a, b$ deux réels strictement positifs et soit $x \in \R$.
\begin{enumerate}[(i)]
\item $\log(a \times b) = \log(a) + \log(b)$.
\item $\log(a^x)=x \log(a)$
\item $\log(\frac1{b})= - \log(b)$ et, plus généralement, $\log \left ( \frac{a}{b} \right ) = \log(a) - \log(b)$.
\end{enumerate}
\end{prop}

\begin{proof}
\begin{enumerate}[(i)]
\item On sait que $10^{a+b} = 10^{a} \times 10^b$, on en déduit le résultat.
\item On démontre le résultat pour $n \in \N$ : 
\begin{align*}
\log(a^2) & = \log(a \times a) & = \log(a) + \log(a) = 2 \log(a) \\
\log(a^3) & = \log(a^2 \times a) & = 2 \log(a) + \log(a) \\
\dots
\end{align*}
En utilisant le point (iii), on étend le résultat à $x \in \Q$. On admet le résultat pour le cas général, $x \in \R$.
\item On sait que $10^{-b} = \dfrac1{10^b}$, on en déduit le résultat. On utilise ensuite le résultat (ii) pour démontrer le cas général.
\end{enumerate}
\end{proof}

\section{Changement de base}

\begin{defn}
Soit $b$ un réel strictement positif et différent de 1 et $a$ un réel strictement positif. On appelle \textbf{logarithme de base $b$} de $a$ l'unique solution, dans $\R$, de l'équation 
\[ b^x = a . \]
On note cette solution $\log_b(a)$.
\end{defn}

\begin{rmq}
Avec cette définition, on peut définir la \textbf{fonction} logarithme de base $b$.
\end{rmq}

\begin{defn}[Fonction logarithme décimal]
Pour $b>0, b \neq 1$, on définit la fonction logarithme de base $b$ sur l'intervalle $]0, \infty[$ par :
\[ \begin{array}{c c c c} \log_b : & ]0, \infty[ & \to & \R \\ & a & \mapsto & \log_b(a) \end{array} \]
\end{defn}

\begin{rmq}
La fonction logarithme de base $b$ est toujours égale à 1 en sa base :
\[ \forall b>0, b \neq 1, \log_b(b) = 1. \]
\end{rmq}

\begin{prop}
Pour tout $b>0, b \neq 1$, les fonctions $\log$ et $\log_b$ sont proportionnelles. Précisément, on a :
\[ \log_b(a) = \dfrac{\log(a)}{\log(b)}. \]
\end{prop}

\begin{proof}
Il suffit de remarquer que :
\[ b^x = a \equi x = \dfrac{\log(a)}{\log(b)} . \]
Comme, par ailleurs, $\log_b(a)$ est l'unique solution de cette équation, on a bien :
\[ \log_b(a) = \dfrac{\log(a)}{\log(7)} . \]
\end{proof}

\begin{coro}
Pour tout $b>0, b \neq 1$, la fonction logarithme de base $b$ est continue. Elle est de plus :
\begin{enumerate}[(i)]
\item croissante si $b > 1$ ;
\item décroissante si $0<b<1$.
\end{enumerate}
\end{coro}

\begin{proof}
Il suffit de remarquer que $\log(b) > 0 \equi b > 1$.
\end{proof}

\begin{coro}
La fonction logarithme de base $b$ vérifie les mêmes propriétés algébriques que la fonction logarithme décimale.
\end{coro}

\chapter{Fonctions exponentielles}

\section{Introduction}

Considérons un investissement financier de 1~000 \euro~ placé sur un compte rémunéré mensuellement à hauteur de $20 \%$.
On peut modéliser ce placement financier à l'aide d'une suite géométrique $(u_n)$ de terme général 
\[ u_n=1000 \cdot (1,2)^n . \]
Le terme de rang $n$ de la suite nous donner l'argent sur le compte après $n$ mois.
Néanmoins, on peut légitimement supposer que l'évolution du montant placé sur le compte n'évolue pas uniquement mensuellement,
après tout, les valeurs boursières sont actualisées à des temporalités très fines.
On aimerait donc connaître l'évolution du capital investi après une durée quelconque, éventuellement très précise (quelque secondes ?)
ce qui certes, d'un point de vue pratique, n'a pas vraiment de sens, mais ce qui titille néanmoins notre curiosité mathématiques. \newline

Une première idée serait de placer dans un repère orthonormée les points $(n, u_n)$ pour un grand nombre de valeurs de $n \in \N$,
puis trouver la courbe continue reliant ses points. 

\begin{center}
\begin{tikzpicture}[>=stealth, scale=1]
	\begin{axis}[xmin = -1, xmax= 9, ymin=900, ymax=4500, axis x line=middle, axis y line=middle, axis line style=<->,
	xlabel={}, ylabel={}, grid=both, grid style = {opacity=.5}]]
	\addplot [only marks] table {
		0 1000
		1 1200
		2 1440
		3 1728
		4 2074
		5 2488
		6 2986
		7 3583
		8 4300
	};
	\addplot[black, thin, domain =0:8] {1000*pow(1.2,x)};
	\end{axis}
\end{tikzpicture}
\end{center}

Par lecture graphique, on peut alors approximer la valeur du capital pour des temps arbitraires. 
Mais cette solution reste insatisfaisante (elle ne donne qu'une approximation). 
Cependant, le graphique nous donne une autre idée : la suite $(u_n)$ est une fonction définie pour des valeurs entières (au mois près),
on pourrait donc prolonger cette fonction pour toutes les valeurs réelles. 
Cette nouvelle fonction prolongée répondrait alors à notre interrogation initiale.

\begin{defn}[Fonctions exponentielles]
Soit $a$ un réel strictement positif. La fonction qui, à tout réel $x$ associe $a^x$, est définie de la manière suivante :
\begin{itemize}
\item
Sur $[0, \infty[$, $f$ est le prolongement à des valeurs non entières positives de la suite géométrique $(a^n)$, où $n$ appartient à $\N$ ;
\item
Sur $]- \infty, 0], a^x = \frac{1}{a^{-x}}$.
\end{itemize}
Cette fonction est appelée \textbf{fonction exponentielle de base $a$}.
\end{defn}

\begin{rmq}
Pour tout réel $a>0$, on a $a^0=1$.
\end{rmq}

\section{Propriétés}

\begin{prop}[Sens de variation]
Soit $a$ un réel strictement positif. La fonction exponentielle de base a est :
\begin{enumerate}[(1)]
\item
Strictement décroissante si $a < 1$.
\item
Constante si $a = 1$.
\item
Strictement croissante si $a > 1$.
\end{enumerate}
\end{prop}

\begin{proof}
Comme la fonction exponentielle de base $a$ est un prolongement de la suite $(a^n)$ sur $]0,\infty[$, 
on va étudier le sens de variation de cette suite.
On remarque que $a^{n+1}-a^n = a^n(a-1)$. Comme $a>0, a^n > 0$ pour tout $n \in \N$.
Donc, le sens de variation de la suite est donné par le signe de $a-1$. 
On a bien le résultat attendu sur $]0,\infty[$. \newline

Sur $]-\infty, 0]$, on notera que la fonction exponentielle de base $a$ 
a un sens de variation inverse à celui de la suite $(a^n) = \frac{1}{a^{-n}}$ (plus $n$ est grand plus on $x$ sse ``rapproche'' de $-\infty$).
On remarque que :
\[ a^{n+1}-a^n = \dfrac1{a^{-(n+1)}} - \dfrac1{a^{-n}} = \dfrac{1-a}{a^{-n+1}} . \]
Donc, le sens de variation de la suite est donné par le signe de $1-a$, on a bien le résultat attendu sur $]0,\infty[$.
\end{proof}

\begin{coro}
Soit $a$ un réel strictement, et $x,y$ deux réels quelconques. On a l'équivalence :
\[ x = y \iff a^x = a^y . \]
\end{coro}

\begin{coro}
Soit $a$ un réel strictement positif et $k$ un réel non nul.
\begin{enumerate}[(1)]
\item
Si $k>0$, la fonction $x \mapsto k a^x$ a le même sens de variation que la fonction exponentielle de base $a$.
\item
Si $k<0$, la fonction $x \mapsto k a^x$ a le sens de variation contraire de la fonction exponentielle de base $a$.
\end{enumerate}
\end{coro}

\begin{prop}[Propriétés algébriques]
Soit $a$ un réel strictement positif, $x$ et $y$ deux réels quelconques et $n$ un entier. On a :
\begin{enumerate}[(i)]
\item
$a^{x+y} = a^x a^y$ ;
\item
$a^{-x} = \frac1{a^x}$ ;
\item
$a^{x-y} = \frac{a^x}{a^y}$ ;
\item
$a^{nx} = (a^x)^n$.
\end{enumerate}
\end{prop}

\section{Application aux évolutions successives}

\begin{prop}
On considère une quantité subissant $n$ évolutions successives. 
On note $p_i$ la valeur initiale de cette quantité et $p_f$ la valeur finale de cette quantité. \newline
Le taux d'évolution totale de cette quantité est donné par $(1+T) = \frac{p_f}{p_i}$.
Le taux d'évolution moyen de cette quantité est solution de l'équation :
\[ (1+t)^n = \frac{p_f}{p_i} \equi t = \left ( \frac{p_f}{p_i} \right ) ^{1/n} - 1 . \]
Pour calculer ce taux moyen, on doit donc évaluer la fonction $x \mapsto \left ( \frac{p_f}{p_i} \right ) ^{x}$ en $\frac1{n}$.
\end{prop}

\chapter{Fonction inverse et polynômes}

\section{Définition}

\begin{rapl}
Soit $x$ un nombre réel \textbf{non nul}. L'inverse multiplicatif de $x$ est le nombre $y$ tel que :
\[ x \cdot y = y \cdot x = 1 . \]
Ce nombre est égal à $\frac1{x}$.
\end{rapl}

\begin{nota}
Dans la suite de ce cours, on notera $\Ret$ l'ensemble des nombres réels privé de 0.
\end{nota}

\begin{defn}[Fonction inverse]
La fonction inverse est la fonction définie sur $\Ret$ par :
\[ \begin{array}{c c c}
\Ret & \to & \R \\
x & \mapsto & \dfrac1{x}
\end{array} \] 
C'est donc la fonction qui, à un réel non nul, associe son inverse multiplicatif.
\end{defn}

\begin{rmq}
la courbe représentative de la fonction inverse est donnée ci-dessous :

\begin{center}
\begin{tikzpicture}[>=stealth, scale=1]
\begin{axis}%
    [
        grid=major,  
        x=5mm,
        y=5mm,
        xtick={-10,-9,...,10},   
        xmin=-10,
        xmax=10,
        xlabel={\tiny $x$},
        axis x line=middle,
        ytick={-10,-9,...,10},
        tick label style={font=\tiny},
        ymin=-10,
        ymax=10,
        ylabel={\tiny $1/x$},
        axis y line=middle,
        no markers,
        samples=100,
        domain=-10:10,
        restrict y to domain=-20:20
    ]
    \addplot[blue, thick,samples=400] (x,{1/x}) ;
\end{axis} 
\end{tikzpicture}
\end{center}

\end{rmq}

\section{Limite et dérivation}

\begin{defn}[Limite]
On considère une fonction $f$ définie sur un intervalle de la forme $]a,b[$ avec $a$ et $b$ éventuellements infinis. \newline

On appelle limite de $f$ en $a$ (ou en $b$), la valeur de la fonction $f$ au voisinage de $a$ (ou de $b$, 
c'est-à-dire au voisinage des bornes de son domaine de définition).
Cette limite peut éventuellement être infinie.
\end{defn}

\begin{ex}
On considère la fonction définie sur $\R$ par  $id : x \mapsto x$. La limite de cette fonction en $\infty$ est égale à $\infty$.
\end{ex}

\begin{prop}
La fonction inverse est définie sur $\R$ privé de 0, autrement dit sur l'union des intervalles $]-\infty ; 0 [$ et $]0, \infty[$.
La fonction inverse admet les limites suivantes :

\begin{enumerate}[(i)]
\item 0 en $-\infty$ ;
\item $-\infty$ \textbf{à gauche} de 0 ;
\item $\infty$ \textbf{à droite} de 0 ;
\item 0 en $\infty$.
\end{enumerate}

\end{prop}

\begin{nomen}
On dit que la fonction inverse admet la droite $y=0$ comme \textbf{asymptote} en $\pm \infty$. 
De façon intuitive, cela signifie que la fonction inverse se rapproche de plus en plus de la fonction nulle 
à mesure que l'on se rapproche de l'infini (peu importe le signe).
\end{nomen}

\begin{rmq}
La fonction inverse admet des limites infinies en 0. Ces limites dépendent du signe considéré 
($\infty$ à droite de 0 et l'opposé à gauche). 
Intuitivement, cela signifie que la fonction inverse devient très grande à mesure que l'on s'approche de 0 par la droite.
A l'opposé, la fonction inverse devient très petite à mesure que l'on s'approche de 0 par la gauche.
\end{rmq}

\begin{rapl}[Dérivée en un point]
La dérivée d'une fonction en un point $x$ de son domaine de définition correspond à la limite de son taux de variation en ce point.
Autrement dit, la dérivée de $f$ en $x$ correspond à :
\[ \lim_{h \to 0} \dfrac{f(x+h)-f(x)}{h} . \]
Cette limite peut éventuellment être infinie et, dans ce cas, on dit que la fonction n'est pas dérivable en $x$.
\end{rapl}

\begin{rmq}
La dérivée d'une fonction $f$ en $x$ correspond à une ``variation instantannée'', 
c'est-à-dire à la variation de la valeur de $f(x)$ pour une variation arbitrairement petite autour de $x$.
En conséquence, une dérivée positive au voisinage de $x$ indique une fonction localement croissante. 
A l'inverse, une dérivée négative au voisinage de $x$ indique une fonction localement décroissante. 
\end{rmq}

\begin{rapl}[Fonction dérivée]
Soit $f$ une fonction dérivable sur un intervalle $I$. On appelle fonction dérivée de $f$, que l'on note $f'$, la fonction qui,
à un point $x$ de $I$ associe la dérivée de $f$ au point $x$.
\[ \begin{array}{c c c c}
f' : & I & \to & \R \\
& x & \mapsto & \lim_{h \to 0} \dfrac{f(x+h)-f(x)}{h} 
\end{array} \]
\end{rapl}

\begin{prop}[Dérivée et sens de variation]
Soit $f$ une fonction numérique et soit $I$ un intervalle inclus dans le domaine de définition de $f$, 
tel que la fonction dérivée $f'$ est de signe constant sur $I$. Alors :
\begin{enumerate}[(i)]
\item si $f' < 0$ sur $I$ alors $f$ est strictement décroissante sur $I$ ;
\item si $f'=0$ sur $I$ alors $f$ est constante sur $I$ ;
\item si $f' > 0$ sur $I$ alors $f$ est strictement croissante sur $I$.
\end{enumerate}
\end{prop}

\begin{rmq}
Si $f' \geq 0$ (respectivement $f' \leq 0$) alors $f$ est simplement croissante (respectivement décroissante) sur $I$.
\end{rmq}

\begin{prop}[Derivée de la fonction inverse]
La fonction inverse admet pour dérivée la fonction suivante :
\[ \begin{array}{c c c}
\Ret & \to & \R \\
x & \mapsto & - \dfrac1{x^2}
\end{array} \] 
\end{prop}

\begin{proof}
On a :
\[ \dfrac{f(x+h)-f(x)}{h} = \dfrac{\frac{1}{x+h}-\frac1{x}}{h} = - \dfrac1{x(x+h)} . \]
On obtient le résultat par passage à la limite.
\end{proof}

\begin{coro}[Sens de variation de la fonction inverse]
La fonction inverse est strictement décroissante sur tout son ensemble de définition.
\end{coro}

\section{Fonctions polynômes}

\begin{defn}[Fonction polynôme de degré 3]
Soit $n \in \N$ et $a, b, c, d$ des réels. On appelle fonction polynôme de degré au plus 3, une fonction de la forme :
\[ \begin{array}{c c c}
\R & \to & \R \\
x & \mapsto & ax^3 + bx^2 +cx + d
\end{array} \]
Le degré de la fonction polynôme est défini de la façon suivante :
\begin{enumerate}[(i)]
\item si $a \neq 0$, alors le degré est égal à 3 ;
\item si $a = 0$ et $b \neq 0$ alos le degré est égal à 2 ;
\item si $a=b=0$ et $c \neq 0$ alors le degré est égal à 1 :
\item si $a=b=c=0$ alors le degré est égal à 0.
\item si $a=b=c=d=0$ alors le degré est égal à $-\infty$.
\end{enumerate}
\end{defn}

\begin{rmq}
On donne ci-dessous les représentations graphiques des fonctions : 
$p_1 : x \mapsto x$, $p_2 : x \mapsto x^2$ et $p_3 : x \mapsto x^3$.
Ce sont les polyômes de degrés respectifs 1, 2 et 3 les plus simples.
\begin{center}
\begin{tikzpicture}[>=stealth, scale=1]
\begin{axis}%
    [
        grid=major,  
        x=5mm,
        y=5mm,
        xtick={-10,-9,...,10},   
        xmin=-10,
        xmax=10,
        xlabel={\tiny $x$},
        axis x line=middle,
        ytick={-10,-9,...,10},
        tick label style={font=\tiny},
        ymin=-10,
        ymax=10,
        ylabel={\tiny $p$},
        axis y line=middle,
        no markers,
        samples=100,
        domain=-10:10,
        restrict y to domain=-20:20
    ]
    \addplot[blue, thick,samples=400] (x,{x}) ;
    \node at (axis cs:8,7){\color{blue} \small $p_1$};
    \addplot[green, thick,samples=400] (x,{x^2}) ;
    \node at (axis cs:3,6){\color{green} \small $p_2$};
    \addplot[red, thick,samples=400] (x,{x^3}) ;
    \node at (axis cs:-3,-7){\color{red} \small $p_3$};
\end{axis} 
\end{tikzpicture}
\end{center}
\end{rmq}

\begin{prop}[Dérivation et polynôme]
On considère la fonction polynôme $P$ définie par :
\[ \begin{array}{c c c c}
P : &\R & \to & \R \\
& x & \mapsto & ax^3 + bx^2 +cx + d
\end{array} \]
La fonction dérivée de $P$ est donnée par :
\[ \begin{array}{c c c c}
P' :  &\R & \to & \R \\
& x & \mapsto & 3ax^2 + 2bx +c
\end{array} \]
\end{prop}

\begin{proof}
On a :
\begin{align*}
P(x+h)-P(x) & = a(x+h)^3 + b(x+h)^2 + c(x+h) + d -ax^3 - bx^2 -cx -d \\
& = ax^3 + 3ax^2h + 3axh^2 + ah^3 + bx^2 + 2bxh + bh^2 +cx + ch \\
&  -ax^3 - bx^2 -cx \\
& = 3ax^2h + 3axh^2 + ah^3 + 2bxh + bh^2 + ch 
\end{align*}
Donc :
\[ \dfrac{P(x+h)-P(x)}{h} = 3ax^2 + 3axh + ah^2 + 2bx + bh + c \]
En faisant tendre $h \to 0$, on retrouve bien la formule attendue.
\end{proof}

\begin{ex}
On considère la fonction polynôme $P$ définie sur $\R$ par :
\[ P(x) = 3x^3 + 6x^2 + 2 . \]
Son degré est égal à 2 et sa fonction dérivée est définie par :
\[ P'(x) = 9x^2 + 12x  . \]
On remarque que le degré de la fonction dérivée est égal à 2.
\end{ex}

\begin{thm}[Linéarité de la dérivation]
Soient $f, g$ deux fonctions dérivables sur un intervalle $I$ et $\lambda, \mu$ deux réels, on a :
\[ (\lambda f + \mu g)' = \lambda f' + \mu g' , \]
sur $I$.
\end{thm}

\begin{proof}
Il suffit de remarquer que :
\[ (\lambda f+\mu g)(x+h) = \lambda f(x+h) - \mu g(x+h) , \]
et de même pour $(\lambda f+\mu g)(x)$. On écrit donc :
\[ \dfrac{ (\lambda f+\mu g)(x+h) - (\lambda f+\mu g)(x)}{h} = \lambda \dfrac{ f(x+h) - f(x) }{h} - \mu \dfrac{ g(x+h) - g(x) }{h} \]
puis on obtient le résultat en faisant tendre $h$ vers 0.
\end{proof}

\begin{ex}
On considère la fonction $f$ définie sur $\Ret$ par $f(x) = 3x^2 + 7x + \frac{8}{x}$. En définissant $f_1$ et $f_2$ sur $\Ret$ par :
\[ f_1(x) = 3x^2 + 7x  \quad \text{et} \quad f_2(x) = \frac{8}{x} \]
on a, en utilisant le théorème précédent :
\[ f'(x) = 6x + 7 - \frac{8}{x^2} . \]
\end{ex}

\end{document}