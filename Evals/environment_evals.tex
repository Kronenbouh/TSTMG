\usepackage[answerdelayed, lastexercise]{exercise}
\usepackage{ifthen}

\renewcommand{\ExerciseHeader}{
	\textbf{
		Exercice \theExercise.
	}
	\ifnum\ExerciseDifficulty=0
	\else
		(\theExerciseDifficulty)
	\fi
}

\newcommand{\exe}[4]{
	\begin{Exercise}[title=#1, label=#3]
		\if\relax\detokenize\expandafter{\ExerciseTitle}\relax
		%\marginpar{[Bonus]}
		\else
		\marginpar{\mbox{[\quad/\ExerciseTitle]}}
		\fi
		#2
	\end{Exercise}
	\begin{Answer}[ref=#3]
		#4
	\end{Answer}
}

\renewcommand{\DifficultyMarker}{$\star$}
\renewcommand{\AnswerHeader}{
	% if exercise title is "1" then announce new chapter
	\if\ExerciseTitle1
		\hrule\vspace{1cm}
		\LARGE
		\textbf{Exercices du chapitre \thechapter}\newline\newline
	\fi
	
	\centerline{\textbf{
	Exercice \ExerciseHeaderNB
	}}
}

\newcommand{\exemulticols}[5]{
	\begin{multicols}{2}
	\begin{Exercise}[title=#1, label=#4]
		\if\relax\detokenize\expandafter{\ExerciseTitle}\relax
		%\marginnote{[Bonus]}
		\else
		\marginnote{\mbox{[\quad/\ExerciseTitle]}}
		\fi
		#2
	\end{Exercise}
	\columnbreak
		#3
	\end{multicols}
	\begin{Answer}[ref=#4]
		#5
	\end{Answer}
}
