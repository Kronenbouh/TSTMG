% Chargement des paquets

\usepackage{amsmath}
\usepackage{amsthm}
\usepackage{amsfonts}
\usepackage{amssymb}
\usepackage{enumerate}
\usepackage{mathtools}
\usepackage{bbm}
\usepackage{xparse, etoolbox}
\usepackage{enumerate}
\usepackage{mathabx}
\usepackage{minted}
\usepackage[french]{babel}
\usepackage{keytheorems}
\usepackage[theorems]{tcolorbox}
\usepackage{hyperref}

% Environnement

%Chargement des paquets
\usepackage{amsmath}
\usepackage{amsthm}
\usepackage{amsfonts}
\usepackage{amssymb}
\usepackage{enumerate}
\usepackage{mathtools}
\usepackage{bbm}
\usepackage{xparse, etoolbox}
\usepackage{enumerate}
\usepackage{mathabx}
\usepackage{minted}
\usepackage[french]{babel}
\usepackage{keytheorems}
\usepackage[theorems]{tcolorbox}
\usepackage{hyperref}
\usepackage[framemethod=tikz]{mdframed}

%Ensembles de nombres
\newcommand{\N} {\mathbb{N}}
\newcommand{\Ne}{\N^\ast}
\newcommand{\Z} {\mathbb{Z}}
\newcommand{\D} {\mathbb{D}}
\newcommand{\Q} {\mathbb{Q}}
\newcommand{\R} {\mathbb{R}}
\newcommand{\Rb}{\overline{\mathbb{R}}}
\newcommand{\Rp}{\R_+}
\newcommand{\Rm}{\R_-}
\newcommand{\K} {\mathbb{K}}
\newcommand{\Cx}{\mathbb{C}}

%Opérateurs
\newcommand{\equi}{\Leftrightarrow}

%Normes
\DeclarePairedDelimiter\abs{\lvert}{\rvert}

%tikz
\usepackage{tikz, pgfplots}
\usetikzlibrary{positioning}
\usetikzlibrary{shapes.geometric}
\usetikzlibrary{positioning}
\usetikzlibrary {angles}
\usepackage{tkz-euclide}

\tikzset{
dot/.style = {circle, fill=#1, minimum size=5pt,
              inner sep=0pt, outer sep=0pt},
dot/.default = black % size of the circle diameter
}

 % for braces
\usetikzlibrary{decorations.pathreplacing}
% for hashing area
\usetikzlibrary{patterns}
% tableaux var, signe
% source https://www.sqlpac.com/fr/documents/latex-package-tkz-tab-tikz-tableaux-de-signes-et-de-variations-de-fonctions.html
\usepackage{tkz-tab}

\tikzset{
	every node/.style = {font=\Large}
}

\tikzset{
	every axis/.style = {clip=true, grid style = {opacity=.5}}
}

%Interface théorème
\renewcommand*{\proofname}{Démonstration}

\theoremstyle{definition}
\newtheorem*{nota}{Notation}

\theoremstyle{definition}
\newtheorem*{conv}{Convention}

\theoremstyle{definition}
\newtheorem{ex}{Exemple}[section]

\theoremstyle{remark}
\newtheorem{rmq}{Remarque}

\theoremstyle{definition}
\newtheorem*{idea}{Idée}

%styles pour théorèmes
\newkeytheoremstyle{tcb-thm}
  {
    headpunct={},
    notebraces={}{},
    noteseparator={ : },
    notefont=\bfseries,
    bodyfont=\slshape,
    tcolorbox={
      arc=0mm,
      colback=blue!5!white,
      colframe=blue!50!black,
      },
  }
  
\newkeytheorem{thm}[
  name=Théorème,
  parent=section,
  style=tcb-thm,
  ]  
  
\newkeytheoremstyle{tcb-prop}
  {
    headpunct={},
    notebraces={}{},
    noteseparator={ : },
    notefont=\bfseries,
    bodyfont=\slshape,
    tcolorbox={
      arc=0mm,
      colback=blue!5!white,
      colframe=blue!75!black,
      },
  }
 
\newkeytheorem{prop}[
  name=Proposition,
  parent=section,
  style=tcb-prop,
  ] 
  
\newkeytheorem{coro}[
  name=Corollaire,
  parent=section,
  style=tcb-prop,
  ]
  
\newkeytheoremstyle{tcb-lem}
  {
    headpunct={},
    notebraces={}{},
    noteseparator={ : },
    notefont=\bfseries,
    bodyfont=\slshape,
    tcolorbox={
      arc=0mm,
      colback=blue!5!white,
      colframe=blue!100!black,
      },
  }
  
\newkeytheorem{lem}[
  name=Lemme,
  parent=section,
  style=tcb-lem,
  ]
 
  
\newkeytheoremstyle{tcb-def}
  {
    headpunct={},
    notebraces={}{},
    noteseparator={ : },
    notefont=\bfseries,
    bodyfont=\slshape,
    tcolorbox={
      arc=0mm,
      colback=orange!5!white,
      colframe=orange!75!black,
      },
  }
  
\newkeytheorem{defn}[
  name=Définition,
  parent=section,
  style=tcb-def,
  ]
 
\newkeytheorem{rapl}[
  name=Rappel,
  style=tcb-def,
  ]
  
\newkeytheoremstyle{tcb-exo}
  {
    headpunct={},
    notebraces={}{},
    noteseparator={ : },
    notefont=\bfseries,
    bodyfont=\slshape,
    tcolorbox={
      arc=0mm,
      colframe=black,
      colback=white,
      },
  }
  
\newkeytheorem{exo}[
  name=Exercice,
  style=tcb-exo,
  ]
  
\surroundwithmdframed[
	hidealllines=true,
	leftline=true,
	innerleftmargin=10pt,
	innerrightmargin=10pt,
	innertopmargin=-4pt,
	nobreak=true,
]{proof}

% Commandes perso

%Ensembles de nombres
\newcommand{\N} {\mathbb{N}}
\newcommand{\Ne}{\N^\ast}
\newcommand{\Z} {\mathbb{Z}}
\newcommand{\D} {\mathbb{D}}
\newcommand{\Q} {\mathbb{Q}}
\newcommand{\R} {\mathbb{R}}
\newcommand{\Rb}{\overline{\mathbb{R}}}
\newcommand{\Rp}{\R_+}
\newcommand{\Rm}{\R_-}
\newcommand{\K} {\mathbb{K}}
\newcommand{\Cx}{\mathbb{C}}

%Opérateurs
\newcommand{\equi}{\Leftrightarrow}

%Normes
\DeclarePairedDelimiter\abs{\lvert}{\rvert}

%Commande d'exo
\newcommand{\exe}[4]{
	\begin{Exercise}[title=#1, label=#3]
		\marginpar{\mbox{\scriptsize(solution p.\pageref{\ExerciseLabel-Answer})}}
		#2
	\end{Exercise}
	\begin{Answer}[ref=#3]
		#4
	\end{Answer}
}

%tikz
\usepackage{tikz, pgfplots}
\usetikzlibrary{positioning}
\usetikzlibrary{shapes.geometric}
\usetikzlibrary{positioning}
\usetikzlibrary {angles}
\usepackage{tkz-euclide}

\tikzset{
dot/.style = {circle, fill=#1, minimum size=5pt,
              inner sep=0pt, outer sep=0pt},
dot/.default = black % size of the circle diameter
}

 % for braces
\usetikzlibrary{decorations.pathreplacing}
% for hashing area
\usetikzlibrary{patterns}
% tableaux var, signe
% source https://www.sqlpac.com/fr/documents/latex-package-tkz-tab-tikz-tableaux-de-signes-et-de-variations-de-fonctions.html
\usepackage{tkz-tab}

\tikzset{
	every node/.style = {font=\Large}
}

\tikzset{
	every axis/.style = {clip=true, grid style = {opacity=.5}}
}
\usepackage{makecell} % commande \thead, dans l'exo 1
\usepackage{eurosym} % pour le symbole euro propre

\begin{document}

\reversemarginpar

\pagestyle{fancy}
\fancyhead[L]{Terminale STMG}
\fancyhead[C]{\textbf{DS n°3a — fonctions exponentielles}}
\fancyhead[R]{\today}

\null\vspace{-30pt}
Nom / Prénom : \\

Consignes particulières : 
\begin{itemize}[label=$\bullet$]
	\item 
	La calculatrice est {autorisée}.
	\item 
	L'exercice \ref{exe:1a} peut être fait entièrement sur la feuille d'évaluation. Écrire son nom avant de rendre le sujet pour qu'il soit corrigé.
	\item 
	Toute trace de recherche est prise en compte.
%	\item 
%	L'abbréviation $\tq$ signifie ``tel que''.
\end{itemize}

\hrule

\exe{6}{
	Vrai ou faux ? Cocher la case correspondante (aucune justification n'est attendue). \\
	\vspace{-20pt}
	\begin{center}
	\begin{tabular}{c c c}
		\hspace{10cm} & Vrai & Faux \\
		La fonction exponentielle de base $a$ est croissante & \resizebox{10pt}{!}{$\square$}  & \resizebox{10pt}{!}{$\square$}   \\
		Si $x=y$ alors $a^x=a^y$ (pour $x, y \in \R$ et $a>0$) & \resizebox{10pt}{!}{$\square$}  & \resizebox{10pt}{!}{$\square$}  \\
		Pour tout $a \in \R, a^0=1$ & \resizebox{10pt}{!}{$\square$}  & \resizebox{10pt}{!}{$\square$}   \\
		$a^x + a^y = a^{x + y}$ & \resizebox{10pt}{!}{$\square$}  & \resizebox{10pt}{!}{$\square$}  \\
		Si $a<0$, la fonction exponentielle de base $a$ est négative & \resizebox{10pt}{!}{$\square$}  & \resizebox{10pt}{!}{$\square$}  \\
		$\left (\dfrac{a^x}{a^x} \right )^{70} = 1^3$ & \resizebox{10pt}{!}{$\square$}  & \resizebox{10pt}{!}{$\square$}  \\
	\end{tabular}
	\end{center}
}{exe:1a}{
Les réponses sont, dans l'ordre : faux, vrai, vrai, faux, faux, vrai.
}

\exe{4}{
Mettre les expressions ci-dessous sous la forme $a^x$.
\begin{multicols}{2}
\begin{enumerate}[label=(\alph*), itemsep=1ex]
\item $a^{3,5} \times a^{3,8}$
\item $\dfrac{a^{5,3}}{a^{6,1}} \times a^{4,9}$
\item $((a^{1,2})^2)^3 \times \dfrac{a^{1,7}}{a^{8,3}}$
\item $\dfrac{1}{a^{4,2} \times a^{2,8}}$
\end{enumerate}
\end{multicols}
}{exe:2a}{
\begin{multicols}{2}
\begin{enumerate}
\item $a^{3,5} \times a^{3,8} = a^{7,3}$
\item $\dfrac{a^{5,3}}{a^{6,1}} \times a^{4,9}= a^{4,1}$
\item $((a^{1,2})^2)^3 \times \dfrac{a^{1,7}}{a^{8,3}} = a^{0,6}$
\item $\dfrac{1}{a^{4,2} \times a^{2,8}} = a^{-7}$
\end{enumerate}
\end{multicols}
}

\exe{4}{
Déterminer le sens de variation des fonctions ci-dessous. Justifier brièvement.
\begin{multicols}{2}
\begin{enumerate}[label=(\alph*), itemsep=1ex]
\item $x \mapsto -\dfrac1{10} \times \left ( \dfrac78 \right )^x$
\item $x \mapsto -0,001 \times (232 )^x$
\item $x \mapsto 42 \times \left ( \dfrac{10}{11} \right )^x$
\item $x \mapsto \dfrac34 \times \left ( 1,05 \right )^x$
\end{enumerate}
\end{multicols}
}{exe:3a}{
\begin{multicols}{2}
\begin{enumerate}
\item croissante	
\item décroissante
\item décroissante
\item croissante
\end{enumerate}
\end{multicols}
}

\exe{3}{
En 1996, en France, le salaire annuel net moyen était de 18~461 \euro~ courants (sans correction de l'inflation).
En 2020, le salaire annuel net moyen était de 29~652  \euro~ courants.
\begin{enumerate}
\item Calculer le taux d'évolution global puis \textbf{annuel moyen} du salaire annuel moyen en France sur la période considérée.
\item Proposer une estimation du salaire annuel moyen en 2026.
\item Proposer une estimation du salaire annuel moyen en 1993.
\end{enumerate}
Question bonus : à partir de quelle année le salaire annuel moyen va-t-il dépasser les 50~000 \euro ?
}{exe:4a}{
\begin{enumerate}
\item Taux d'évolution globale : $T = \left ( \frac{29~652}{18~461} \right ) - 1 \approx 61 \%$ \newline
Taux d'évolution moyen : $t = \left ( \frac{29~652}{18~461} \right )^{1/24} - 1 \approx 2,0 \%$.
\item On calcule $29~652 \times (1+t)^6 \approx 33~381$ \euro.
\item On calcule $18~461 \times (1+t)^{-3} \approx 17~399$ \euro.
\end{enumerate}
Bonus : on doit résoudre $29~652 \times (1+t)^n \geq 50~000$, en passant au $\log$ on trouve :
\[ n \geq \dfrac{\log \left ( \dfrac{50~000}{29~652} \right)}{\log(1+t)} \approx 26,4 \]
En considérant ce rythme de croissance, le salaire annuel moyen dépassera les 50~000 \euro~ en 2047.
}

\exe{3}{
Une usine augmente annuellement (de de façon constante) sa production de $9 \%$. En 2015, elle produisait 200 unités.
\begin{enumerate}
\item Quel est le taux d'évolution annuel moyen de la production de l'usine ?
\item Estimer la production de l'usine en 2026.
\item Chaque unité produite est vendue 3~000 \euro. En tenant compte de frais fixes de 90~000 \euro~annuels, 
estimer le résultat net de l'entreprise opérant l'usine en 2026.
\end{enumerate}
}{exe:5a}{
\begin{enumerate}
\item Le taux d'évolution annuel moyen de la production de l'usine est de $9 \%$.
\item On calcule $200 \times (1,09)^{11} \approx 516$.
\item Le chiffre d'affaire est de $3000 \times 516 \approx 1~548~000$ \euro~ en 2026, soit un résultat net de $1~458~000$ \euro.
\end{enumerate}
}

%%%%%%%%%%%%

\setcounter{Exercise}{0}

\newpage

\pagestyle{fancy}
\fancyhead[L]{Terminale STMG}
\fancyhead[C]{\textbf{DS n°3b — fonctions exponentielles}}
\fancyhead[R]{\today}

\null\vspace{-30pt}
Nom / Prénom : \\

Consignes particulières : 
\begin{itemize}[label=$\bullet$]
	\item 
	La calculatrice est {autorisée}.
	\item 
	L'exercice \ref{exe:1b} peut être fait entièrement sur la feuille d'évaluation. Écrire son nom avant de rendre le sujet pour qu'il soit corrigé.
	\item 
	Toute trace de recherche est prise en compte.
%	\item 
%	L'abbréviation $\tq$ signifie ``tel que''.
\end{itemize}

\hrule

\exe{6}{
	Vrai ou faux ? Cocher la case correspondante (aucune justification n'est attendue). \\
	\vspace{-20pt}
	\begin{center}
	\begin{tabular}{c c c}
		\hspace{10cm} & Vrai & Faux \\
		Si $x=y$ alors $a^x=a^y$ (pour $x, y \in \R$ et $a>0$) & \resizebox{10pt}{!}{$\square$}  & \resizebox{10pt}{!}{$\square$}  \\
		$a^x + a^y = a^{x + y}$ & \resizebox{10pt}{!}{$\square$}  & \resizebox{10pt}{!}{$\square$}  \\
		La fonction exponentielle de base $a$ est croissante & \resizebox{10pt}{!}{$\square$}  & \resizebox{10pt}{!}{$\square$}   \\
		$\left (\dfrac{a^x}{a^x} \right )^{70} = 1^3$ & \resizebox{10pt}{!}{$\square$}  & \resizebox{10pt}{!}{$\square$}  \\
		Pour tout $a \in \R, a^0=1$ & \resizebox{10pt}{!}{$\square$}  & \resizebox{10pt}{!}{$\square$}   \\
		Si $a<0$, la fonction exponentielle de base $a$ est positive & \resizebox{10pt}{!}{$\square$}  & \resizebox{10pt}{!}{$\square$}  \\
	\end{tabular}
	\end{center}
}{exe:1b}{
Les réponses sont, dans l'ordre : vrai, faux, faux, vrai, vrai, faux.
}

\exe{4}{
Mettre les expressions ci-dessous sous la forme $a^x$.
\begin{multicols}{2}
\begin{enumerate}[label=(\alph*), itemsep=1ex]
\item $a^{5,3} \times a^{3,8}$
\item $\dfrac{a^{3,5}}{a^{6,1}} \times a^{4,9}$
\item $((a^{2,1})^2)^3 \times \dfrac{a^{1,7}}{a^{6,3}}$
\item $\dfrac{1}{a^{4,3} \times a^{2,5}}$
\end{enumerate}
\end{multicols}
}{exe:2b}{
\begin{enumerate}
\item $a^{5,3} \times a^{3,8} = a^{9,1}$
\item $\dfrac{a^{3,5}}{a^{6,1}} \times a^{4,9} = a^{2,3}$
\item $((a^{2,1})^2)^3 \times \dfrac{a^{1,7}}{a^{6,3}} = a^8$
\item $\dfrac{1}{a^{4,3} \times a^{2,5}} = a^{-6,8}$
\end{enumerate}
}

\exe{4}{
Déterminer le sens de variation des fonctions ci-dessous. Justifier brièvement.
\begin{multicols}{2}
\begin{enumerate}[label=(\alph*), itemsep=1ex]
\item $x \mapsto 23 \times \left ( \dfrac{10}{11} \right )^x$
\item $x \mapsto -\dfrac1{10} \times \left ( \dfrac87 \right )^x$
\item $x \mapsto \dfrac34 \times \left ( 1,05 \right )^x$
\item $x \mapsto 0,001 \times (232 )^x$
\end{enumerate}
\end{multicols}
}{exe:3b}{
\begin{multicols}{2}
\begin{enumerate}
\item décroissante
\item décroissante
\item croissante
\item croissante
\end{enumerate}
\end{multicols}
}

\exe{3}{
En 1999, en France, le salaire annuel net moyen était de 19~486 \euro~ courants (sans correction de l'inflation).
En 2022, le salaire annuel net moyen était de 30~721  \euro~ courants.
\begin{enumerate}
\item Calculer le taux d'évolution global puis \textbf{annuel moyen} du salaire annuel moyen en France sur la période considérée.
\item Proposer une estimation du salaire annuel moyen en 2029.
\item Proposer une estimation du salaire annuel moyen en 1996.
\end{enumerate}
Question bonus : à partir de quelle année le salaire annuel moyen va-t-il dépasser les 55~000 \euro ?
}{exe:4b}{
\begin{enumerate}
\item Taux d'évolution globale : $T = \left ( \frac{30~721}{19~486} \right ) - 1 \approx 58 \%$ \newline
Taux d'évolution moyen : $t = \left ( \frac{30~721}{19~486} \right )^{1/23} - 1 \approx 2,0 \%$.
\item On calcule $30~721 \times (1+t)^7 \approx 35~287$ \euro.
\item On calcule $19~486 \times (1+t)^{-3} \approx 18~363$ \euro.
\end{enumerate}
Bonus : on doit résoudre $30~721 \times (1+t)^n \geq 55~000$, en passant au $\log$ on trouve :
\[ n \geq \dfrac{\log \left ( \dfrac{55~000}{30~721} \right)}{\log(1+t)} \approx 29,4 \]
En considérant ce rythme de croissance, le salaire annuel moyen dépassera les 55~000 \euro~ en 2052.
}

\exe{3}{
Une usine augmente annuellement (de de façon constante) sa production de $6 \%$. En 2015, elle produisait 400 unités.
\begin{enumerate}
\item Quel est le taux d'évolution annuel moyen de la production de l'usine ?
\item Estimer la production de l'usine en 2026.
\item Chaque unité produite est vendue 2~000 \euro. En tenant compte de frais fixes de 70~000 \euro~annuels, 
calculer le résultat net de l'entreprise opérant l'usine en 2026.
\end{enumerate}
}{exe:5b}{
\begin{enumerate}
\item Le taux d'évolution annuel moyen de la production de l'usine est de $6 \%$.
\item On calcule $400 \times (1,06)^{11} \approx 759$.
\item Le chiffre d'affaire est de $2000 \times 759 \approx 1~518~000$ \euro~ en 2026, soit un résultat net de $1~448~000$ \euro.
\end{enumerate}
}

%%%%%%%%%%%%

\newpage
\fancyhead[C]{\textbf{Solutions}}
Les solutions sont données dans l'ordre sujet a puis b. \newline
\shipoutAnswer


\end{document}
