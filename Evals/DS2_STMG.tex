% Chargement des paquets

\usepackage{amsmath}
\usepackage{amsthm}
\usepackage{amsfonts}
\usepackage{amssymb}
\usepackage{enumerate}
\usepackage{mathtools}
\usepackage{bbm}
\usepackage{xparse, etoolbox}
\usepackage{enumerate}
\usepackage{mathabx}
\usepackage{minted}
\usepackage[french]{babel}
\usepackage{keytheorems}
\usepackage[theorems]{tcolorbox}
\usepackage{hyperref}

% Environnement

%Chargement des paquets
\usepackage{amsmath}
\usepackage{amsthm}
\usepackage{amsfonts}
\usepackage{amssymb}
\usepackage{enumerate}
\usepackage{mathtools}
\usepackage{bbm}
\usepackage{xparse, etoolbox}
\usepackage{enumerate}
\usepackage{mathabx}
\usepackage{minted}
\usepackage[french]{babel}
\usepackage{keytheorems}
\usepackage[theorems]{tcolorbox}
\usepackage{hyperref}
\usepackage[framemethod=tikz]{mdframed}

%Ensembles de nombres
\newcommand{\N} {\mathbb{N}}
\newcommand{\Ne}{\N^\ast}
\newcommand{\Z} {\mathbb{Z}}
\newcommand{\D} {\mathbb{D}}
\newcommand{\Q} {\mathbb{Q}}
\newcommand{\R} {\mathbb{R}}
\newcommand{\Rb}{\overline{\mathbb{R}}}
\newcommand{\Rp}{\R_+}
\newcommand{\Rm}{\R_-}
\newcommand{\K} {\mathbb{K}}
\newcommand{\Cx}{\mathbb{C}}

%Opérateurs
\newcommand{\equi}{\Leftrightarrow}

%Normes
\DeclarePairedDelimiter\abs{\lvert}{\rvert}

%tikz
\usepackage{tikz, pgfplots}
\usetikzlibrary{positioning}
\usetikzlibrary{shapes.geometric}
\usetikzlibrary{positioning}
\usetikzlibrary {angles}
\usepackage{tkz-euclide}

\tikzset{
dot/.style = {circle, fill=#1, minimum size=5pt,
              inner sep=0pt, outer sep=0pt},
dot/.default = black % size of the circle diameter
}

 % for braces
\usetikzlibrary{decorations.pathreplacing}
% for hashing area
\usetikzlibrary{patterns}
% tableaux var, signe
% source https://www.sqlpac.com/fr/documents/latex-package-tkz-tab-tikz-tableaux-de-signes-et-de-variations-de-fonctions.html
\usepackage{tkz-tab}

\tikzset{
	every node/.style = {font=\Large}
}

\tikzset{
	every axis/.style = {clip=true, grid style = {opacity=.5}}
}

%Interface théorème
\renewcommand*{\proofname}{Démonstration}

\theoremstyle{definition}
\newtheorem*{nota}{Notation}

\theoremstyle{definition}
\newtheorem*{conv}{Convention}

\theoremstyle{definition}
\newtheorem{ex}{Exemple}[section]

\theoremstyle{remark}
\newtheorem{rmq}{Remarque}

\theoremstyle{definition}
\newtheorem*{idea}{Idée}

%styles pour théorèmes
\newkeytheoremstyle{tcb-thm}
  {
    headpunct={},
    notebraces={}{},
    noteseparator={ : },
    notefont=\bfseries,
    bodyfont=\slshape,
    tcolorbox={
      arc=0mm,
      colback=blue!5!white,
      colframe=blue!50!black,
      },
  }
  
\newkeytheorem{thm}[
  name=Théorème,
  parent=section,
  style=tcb-thm,
  ]  
  
\newkeytheoremstyle{tcb-prop}
  {
    headpunct={},
    notebraces={}{},
    noteseparator={ : },
    notefont=\bfseries,
    bodyfont=\slshape,
    tcolorbox={
      arc=0mm,
      colback=blue!5!white,
      colframe=blue!75!black,
      },
  }
 
\newkeytheorem{prop}[
  name=Proposition,
  parent=section,
  style=tcb-prop,
  ] 
  
\newkeytheorem{coro}[
  name=Corollaire,
  parent=section,
  style=tcb-prop,
  ]
  
\newkeytheoremstyle{tcb-lem}
  {
    headpunct={},
    notebraces={}{},
    noteseparator={ : },
    notefont=\bfseries,
    bodyfont=\slshape,
    tcolorbox={
      arc=0mm,
      colback=blue!5!white,
      colframe=blue!100!black,
      },
  }
  
\newkeytheorem{lem}[
  name=Lemme,
  parent=section,
  style=tcb-lem,
  ]
 
  
\newkeytheoremstyle{tcb-def}
  {
    headpunct={},
    notebraces={}{},
    noteseparator={ : },
    notefont=\bfseries,
    bodyfont=\slshape,
    tcolorbox={
      arc=0mm,
      colback=orange!5!white,
      colframe=orange!75!black,
      },
  }
  
\newkeytheorem{defn}[
  name=Définition,
  parent=section,
  style=tcb-def,
  ]
 
\newkeytheorem{rapl}[
  name=Rappel,
  style=tcb-def,
  ]
  
\newkeytheoremstyle{tcb-exo}
  {
    headpunct={},
    notebraces={}{},
    noteseparator={ : },
    notefont=\bfseries,
    bodyfont=\slshape,
    tcolorbox={
      arc=0mm,
      colframe=black,
      colback=white,
      },
  }
  
\newkeytheorem{exo}[
  name=Exercice,
  style=tcb-exo,
  ]
  
\surroundwithmdframed[
	hidealllines=true,
	leftline=true,
	innerleftmargin=10pt,
	innerrightmargin=10pt,
	innertopmargin=-4pt,
	nobreak=true,
]{proof}

% Commandes perso

%Ensembles de nombres
\newcommand{\N} {\mathbb{N}}
\newcommand{\Ne}{\N^\ast}
\newcommand{\Z} {\mathbb{Z}}
\newcommand{\D} {\mathbb{D}}
\newcommand{\Q} {\mathbb{Q}}
\newcommand{\R} {\mathbb{R}}
\newcommand{\Rb}{\overline{\mathbb{R}}}
\newcommand{\Rp}{\R_+}
\newcommand{\Rm}{\R_-}
\newcommand{\K} {\mathbb{K}}
\newcommand{\Cx}{\mathbb{C}}

%Opérateurs
\newcommand{\equi}{\Leftrightarrow}

%Normes
\DeclarePairedDelimiter\abs{\lvert}{\rvert}

%Commande d'exo
\newcommand{\exe}[4]{
	\begin{Exercise}[title=#1, label=#3]
		\marginpar{\mbox{\scriptsize(solution p.\pageref{\ExerciseLabel-Answer})}}
		#2
	\end{Exercise}
	\begin{Answer}[ref=#3]
		#4
	\end{Answer}
}

%tikz
\usepackage{tikz, pgfplots}
\usetikzlibrary{positioning}
\usetikzlibrary{shapes.geometric}
\usetikzlibrary{positioning}
\usetikzlibrary {angles}
\usepackage{tkz-euclide}

\tikzset{
dot/.style = {circle, fill=#1, minimum size=5pt,
              inner sep=0pt, outer sep=0pt},
dot/.default = black % size of the circle diameter
}

 % for braces
\usetikzlibrary{decorations.pathreplacing}
% for hashing area
\usetikzlibrary{patterns}
% tableaux var, signe
% source https://www.sqlpac.com/fr/documents/latex-package-tkz-tab-tikz-tableaux-de-signes-et-de-variations-de-fonctions.html
\usepackage{tkz-tab}

\tikzset{
	every node/.style = {font=\Large}
}

\tikzset{
	every axis/.style = {clip=true, grid style = {opacity=.5}}
}
\usepackage{makecell} % commande \thead, dans l'exo 1
\usepackage{eurosym} % pour le symbole euro propre

\begin{document}

\reversemarginpar

\pagestyle{fancy}
\fancyhead[L]{Terminale STMG}
\fancyhead[C]{\textbf{DS n°2a — logarithme}}
\fancyhead[R]{\today}

\null\vspace{-30pt}
Nom / Prénom : \\

Consignes particulières : 
\begin{itemize}[label=$\bullet$]
	\item 
	La calculatrice est {autorisée}.
	\item 
	L'exercice \ref{exe:1a} peut être fait entièrement sur la feuille d'évaluation. Écrire son nom avant de rendre le sujet pour qu'il soit corrigé.
	\item 
	Toute trace de recherche est prise en compte.
%	\item 
%	L'abbréviation $\tq$ signifie ``tel que''.
\end{itemize}

\hrule

\exe{6}{
	Vrai ou faux ? Cocher la case correspondante (aucune justification n'est attendue). \\
	\vspace{-20pt}
	\begin{center}
	\begin{tabular}{c c c}
		\hspace{10cm} & Vrai & Faux \\
		$\log(-10) = -1$ & $\square$ & $\square$  \\
		Si $9^x < 1000$ alors $x < \frac{3}{\log(9)}$ & $\square$ & $\square$ \\
		La fonction logarithme décimale est croissante & $\square$ & $\square$  \\
		Pour $a >0$ et $b>0, \log(a+b) = \log(a) \times \log(b)$ & $\square$ & $\square$ \\
		Si $0<a<1$ alors $\log(a)<0$ & $\square$ & $\square$ \\
		\thead{Sur un axe logarithmique de graduation 1 cm \\
		la valeur $2~000~000$ est à environ 6 cm} & $\square$ & $\square$ \\
	\end{tabular}
	\end{center}
}{exe:1a}{
Les réponses sont, dans l'ordre : faux, vrai, vrai, faux, vrai, vrai.
}

\exe{4}{
En utilisant les propriétés du cours, justifier les égalités ci-dessous :
\begin{multicols}{2}
\begin{enumerate}[label=(\alph*)]
\item $\log(107) = \log(428) - 2\log(2)$
\item $\log(45) = 2\log(3) + \log(5)$
\item $\log(50) = \log(34) - \log(17) + 2\log(5)$
\item $\log(30) = \log(2)+\log(3)+\log(5)$
\end{enumerate}
\end{multicols}
}{exe:2a}{
\begin{enumerate}
\item $\log(428) - 2\log(2) = \log \left ( \frac{428}{2^2} \right ) = \log(107)$ selon les propriétés (i) et (ii).
\item $2\log(3) + \log(5) = \log(3^2 \times 5) = \log(45)$ selon les propriétés (i) et (ii).
\item $\log(34) - \log(17) + 2\log(5) = \log \left (\frac{34}{17} \times 5^2 \right ) = \log(50)$ selon les propriétés (i), (ii) et (iii).
\item $\log(2)+\log(3)+\log(5) = \log(2 \times 3 \times 5) = \log(30)$ selon la propriété (i).
\end{enumerate}
}

\exe{5}{
Résoudre dans $\R$ les équations suivantes :
\begin{multicols}{2}
\begin{enumerate}[label=(\alph*)]
\item $9^x = 810$
\item $x^{15} = 3000$
\item $17^x < 2040$
\item $9 \leq 5^x < 212$
\item $25 \cdot 6^x = 232$
\end{enumerate}
\end{multicols}
}{exe:3a}{
\begin{multicols}{2}
\begin{enumerate}
\item $x = \frac{\log(810)}{\log(9)}$
\item $x = 3000^{1/15}$
\item $x < \frac{\log(2040)}{\log(17)}$ car $\log(17)>0$
\item $\frac{\log(9)}{\log(5)} \leq x < \frac{\log(212)}{\log(5)}$ car $\log(5)>0$
\item $x = \frac{\log(232)-\log(25)}{\log(6)}$
\end{enumerate}
\end{multicols}
}

\exe{5}{
On considère une population bactérienne dont le taux de croissance est de $3 \%$ par minute.
Une personne infectée par cette bactérie tousse et propulse une population de $1~000$ bactéries sur une surface (une table par exemple).
\begin{enumerate}
\item Peut-on utiliser une suite pour modéliser la croissance de la population bactérienne ? 
Si oui, préciser son type (arithmétique, géométrique, autre ?), son premier terme et sa raison.
\item 
Donner le nombre de bactéries présentes sur la table après 10 minutes.
\item
Si vous deviez représenter la croissance de la population bactérienne dans un repère, 
quel type d'échelle serait adaptée pour l'axe des ordonnées ?
\item 
Donner une estimation de la durée nécessaire pour que la population bactérienne dépasse le million d'individus.
\end{enumerate}
\textit{On rappelle que $1~000~000 = 10^6$.}
}{exe:4a}{
\begin{enumerate}
\item Oui, une suité géométrique de premier terme $1~000$ et de raison $1,03$.
\item Après 10 minutes, il y a $1000 \times (1,03)^{10} \approx 1344$ bactéries sur la table.
\item Une échelle lograithmique (pour l'axe des ordonnées).
\item On cherche $n$ tel que :
\[ 1000 \times (1,03)^n > 10^6 \iff n > \dfrac{3}{\log(1,03)} \approx 233,7 . \]
Il faut donc 234 minutes pour que la population bactérienne dépasse le million d'individus.
\end{enumerate}
}

%%%%%%%%%%%%

\setcounter{Exercise}{0}

\newpage

\pagestyle{fancy}
\fancyhead[L]{Terminale STMG}
\fancyhead[C]{\textbf{DS n°2b — logarithme}}
\fancyhead[R]{\today}

\null\vspace{-30pt}
Nom / Prénom : \\

Consignes particulières : 
\begin{itemize}[label=$\bullet$]
	\item 
	La calculatrice est {autorisée}.
	\item 
	L'exercice \ref{exe:1b} peut être fait entièrement sur la feuille d'évaluation. Écrire son nom avant de rendre le sujet pour qu'il soit corrigé.
	\item 
	Toute trace de recherche est prise en compte.
%	\item 
%	L'abbréviation $\tq$ signifie ``tel que''.
\end{itemize}

\hrule

\exe{6}{
	Vrai ou faux ? Cocher la case correspondante (aucune justification n'est attendue). \\
	\vspace{-20pt}
	\begin{center}
	\begin{tabular}{c c c}
		\hspace{10cm} & Vrai & Faux \\
		Si $0<a<1$ alors $\log(a)<0$ & $\square$ & $\square$ \\
		$\log(-10) = -1$ & $\square$ & $\square$  \\
		\thead{Sur un axe logarithmique de graduation 1 cm \\
		la valeur $2~000~000$ est à environ 6 cm} & $\square$ & $\square$ \\
		Si $9^x < 1000$ alors $x < \frac{3}{\log(9)}$ & $\square$ & $\square$ \\
		La fonction logarithme décimale est croissante & $\square$ & $\square$  \\
		Pour $a >0$ et $b>0, \log(a+b) = \log(a) \times \log(b)$ & $\square$ & $\square$ \\
	\end{tabular}
	\end{center}
}{exe:1b}{
Les réponses sont, dans l'ordre : vrai, faux, vrai, vrai, vrai, faux.
}

\exe{4}{
En utilisant les propriétés du cours, justifier les égalités ci-dessous :
\begin{multicols}{2}
\begin{enumerate}[label=(\alph*)]
\item $\log(30) = \log(2)+\log(3)+\log(5)$
\item $\log(108) = \log(432) - 2\log(2)$
\item $\log(50) = \log(38) - \log(19) + 2\log(5)$
\item $\log(45) = 2\log(3) + \log(5)$
\end{enumerate}
\end{multicols}
}{exe:2b}{
\begin{enumerate}
\item $\log(2)+\log(3)+\log(5) = \log(2 \times 3 \times 5) = \log(30)$ selon la propriété (i).
\item $\log(432) - 2\log(2) = \log \left ( \frac{432}{2^2} \right ) = \log(108)$ selon les propriétés (i) et (ii).
\item $\log(38) - \log(19) + 2\log(5) = \log \left (\frac{38}{19} \times 5^2 \right ) = \log(50)$ selon les propriétés (i), (ii) et (iii).
\item $2\log(3) + \log(5) = \log(3^2 \times 5) = \log(45)$ selon les propriétés (i) et (ii).
\end{enumerate}
}

\exe{5}{
Résoudre dans $\R$ les équations suivantes :
\begin{multicols}{2}
\begin{enumerate}[label=(\alph*)]
\item $9^x = 710$
\item $x^{25} = 3000$
\item $17^x < 2400$
\item $9 \leq 5^x < 221$
\item $25 \cdot 6^x = 2321$
\end{enumerate}
\end{multicols}
}{exe:3b}{
\begin{multicols}{2}
\begin{enumerate}
\item $x = \frac{\log(710)}{\log(9)}$
\item $x = 3000^{1/25}$
\item $x < \frac{\log(2400)}{\log(17)}$ car $\log(17)>0$
\item $\frac{\log(9)}{\log(5)} \leq x < \frac{\log(221)}{\log(5)}$ car $\log(5)>0$
\item $x = \frac{\log(2321)-\log(25)}{\log(6)}$
\end{enumerate}
\end{multicols}
}

\exe{5}{
On considère une population bactérienne dont le taux de croissance est de $5 \%$ par minute.
Une personne infectée par cette bactérie tousse et propulse une population de $100$ bactéries sur une surface (une table par exemple).
\begin{enumerate}
\item Peut-on utiliser une suite pour modéliser la croissance de la population bactérienne ? 
Si oui, préciser son type (arithmétique, géométrique, autre ?), son premier terme et sa raison.
\item 
Donner le nombre de bactéries présentes sur la table après 10 minutes.
\item
Si vous deviez représenter la croissance de la population bactérienne dans un repère, 
quel type d'échelle serait adaptée pour l'axe des ordonnées ?
\item 
Donner une estimation de la durée nécessaire pour que la population bactérienne dépasse le million d'individus.
\end{enumerate}
\textit{On rappelle que $1~000~000 = 10^6$.}
}{exe:4b}{
\begin{enumerate}
\item Oui, une suité géométrique de premier terme $100$ et de raison $1,05$.
\item Après 10 minutes, il y a $100 \times (1,05)^{10} \approx 163$ bactéries sur la table.
\item Une échelle logarithmique (pour l'axe des ordonnées).
\item On cherche $n$ tel que :
\[ 100 \times (1,05)^n > 10^6 \iff n > \dfrac{4}{\log(1,05)} \approx 189 . \]
Il faut donc 189 minutes pour que la population bactérienne dépasse le million d'individus.
\end{enumerate}
}
%%%%%%%%%%%%

\newpage
\fancyhead[C]{\textbf{Solutions}}
Les solutions sont données dans l'ordre sujet a puis b. \newline
\shipoutAnswer


\end{document}
