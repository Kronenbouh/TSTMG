% Chargement des paquets

\usepackage{amsmath}
\usepackage{amsthm}
\usepackage{amsfonts}
\usepackage{amssymb}
\usepackage{enumerate}
\usepackage{mathtools}
\usepackage{bbm}
\usepackage{xparse, etoolbox}
\usepackage{enumerate}
\usepackage{mathabx}
\usepackage{minted}
\usepackage[french]{babel}
\usepackage{keytheorems}
\usepackage[theorems]{tcolorbox}
\usepackage{hyperref}

% Environnement

%Chargement des paquets
\usepackage{amsmath}
\usepackage{amsthm}
\usepackage{amsfonts}
\usepackage{amssymb}
\usepackage{enumerate}
\usepackage{mathtools}
\usepackage{bbm}
\usepackage{xparse, etoolbox}
\usepackage{enumerate}
\usepackage{mathabx}
\usepackage{minted}
\usepackage[french]{babel}
\usepackage{keytheorems}
\usepackage[theorems]{tcolorbox}
\usepackage{hyperref}
\usepackage[framemethod=tikz]{mdframed}

%Ensembles de nombres
\newcommand{\N} {\mathbb{N}}
\newcommand{\Ne}{\N^\ast}
\newcommand{\Z} {\mathbb{Z}}
\newcommand{\D} {\mathbb{D}}
\newcommand{\Q} {\mathbb{Q}}
\newcommand{\R} {\mathbb{R}}
\newcommand{\Rb}{\overline{\mathbb{R}}}
\newcommand{\Rp}{\R_+}
\newcommand{\Rm}{\R_-}
\newcommand{\K} {\mathbb{K}}
\newcommand{\Cx}{\mathbb{C}}

%Opérateurs
\newcommand{\equi}{\Leftrightarrow}

%Normes
\DeclarePairedDelimiter\abs{\lvert}{\rvert}

%tikz
\usepackage{tikz, pgfplots}
\usetikzlibrary{positioning}
\usetikzlibrary{shapes.geometric}
\usetikzlibrary{positioning}
\usetikzlibrary {angles}
\usepackage{tkz-euclide}

\tikzset{
dot/.style = {circle, fill=#1, minimum size=5pt,
              inner sep=0pt, outer sep=0pt},
dot/.default = black % size of the circle diameter
}

 % for braces
\usetikzlibrary{decorations.pathreplacing}
% for hashing area
\usetikzlibrary{patterns}
% tableaux var, signe
% source https://www.sqlpac.com/fr/documents/latex-package-tkz-tab-tikz-tableaux-de-signes-et-de-variations-de-fonctions.html
\usepackage{tkz-tab}

\tikzset{
	every node/.style = {font=\Large}
}

\tikzset{
	every axis/.style = {clip=true, grid style = {opacity=.5}}
}

%Interface théorème
\renewcommand*{\proofname}{Démonstration}

\theoremstyle{definition}
\newtheorem*{nota}{Notation}

\theoremstyle{definition}
\newtheorem*{conv}{Convention}

\theoremstyle{definition}
\newtheorem{ex}{Exemple}[section]

\theoremstyle{remark}
\newtheorem{rmq}{Remarque}

\theoremstyle{definition}
\newtheorem*{idea}{Idée}

%styles pour théorèmes
\newkeytheoremstyle{tcb-thm}
  {
    headpunct={},
    notebraces={}{},
    noteseparator={ : },
    notefont=\bfseries,
    bodyfont=\slshape,
    tcolorbox={
      arc=0mm,
      colback=blue!5!white,
      colframe=blue!50!black,
      },
  }
  
\newkeytheorem{thm}[
  name=Théorème,
  parent=section,
  style=tcb-thm,
  ]  
  
\newkeytheoremstyle{tcb-prop}
  {
    headpunct={},
    notebraces={}{},
    noteseparator={ : },
    notefont=\bfseries,
    bodyfont=\slshape,
    tcolorbox={
      arc=0mm,
      colback=blue!5!white,
      colframe=blue!75!black,
      },
  }
 
\newkeytheorem{prop}[
  name=Proposition,
  parent=section,
  style=tcb-prop,
  ] 
  
\newkeytheorem{coro}[
  name=Corollaire,
  parent=section,
  style=tcb-prop,
  ]
  
\newkeytheoremstyle{tcb-lem}
  {
    headpunct={},
    notebraces={}{},
    noteseparator={ : },
    notefont=\bfseries,
    bodyfont=\slshape,
    tcolorbox={
      arc=0mm,
      colback=blue!5!white,
      colframe=blue!100!black,
      },
  }
  
\newkeytheorem{lem}[
  name=Lemme,
  parent=section,
  style=tcb-lem,
  ]
 
  
\newkeytheoremstyle{tcb-def}
  {
    headpunct={},
    notebraces={}{},
    noteseparator={ : },
    notefont=\bfseries,
    bodyfont=\slshape,
    tcolorbox={
      arc=0mm,
      colback=orange!5!white,
      colframe=orange!75!black,
      },
  }
  
\newkeytheorem{defn}[
  name=Définition,
  parent=section,
  style=tcb-def,
  ]
 
\newkeytheorem{rapl}[
  name=Rappel,
  style=tcb-def,
  ]
  
\newkeytheoremstyle{tcb-exo}
  {
    headpunct={},
    notebraces={}{},
    noteseparator={ : },
    notefont=\bfseries,
    bodyfont=\slshape,
    tcolorbox={
      arc=0mm,
      colframe=black,
      colback=white,
      },
  }
  
\newkeytheorem{exo}[
  name=Exercice,
  style=tcb-exo,
  ]
  
\surroundwithmdframed[
	hidealllines=true,
	leftline=true,
	innerleftmargin=10pt,
	innerrightmargin=10pt,
	innertopmargin=-4pt,
	nobreak=true,
]{proof}

% Commandes perso

%Ensembles de nombres
\newcommand{\N} {\mathbb{N}}
\newcommand{\Ne}{\N^\ast}
\newcommand{\Z} {\mathbb{Z}}
\newcommand{\D} {\mathbb{D}}
\newcommand{\Q} {\mathbb{Q}}
\newcommand{\R} {\mathbb{R}}
\newcommand{\Rb}{\overline{\mathbb{R}}}
\newcommand{\Rp}{\R_+}
\newcommand{\Rm}{\R_-}
\newcommand{\K} {\mathbb{K}}
\newcommand{\Cx}{\mathbb{C}}

%Opérateurs
\newcommand{\equi}{\Leftrightarrow}

%Normes
\DeclarePairedDelimiter\abs{\lvert}{\rvert}

%Commande d'exo
\newcommand{\exe}[4]{
	\begin{Exercise}[title=#1, label=#3]
		\marginpar{\mbox{\scriptsize(solution p.\pageref{\ExerciseLabel-Answer})}}
		#2
	\end{Exercise}
	\begin{Answer}[ref=#3]
		#4
	\end{Answer}
}

%tikz
\usepackage{tikz, pgfplots}
\usetikzlibrary{positioning}
\usetikzlibrary{shapes.geometric}
\usetikzlibrary{positioning}
\usetikzlibrary {angles}
\usepackage{tkz-euclide}

\tikzset{
dot/.style = {circle, fill=#1, minimum size=5pt,
              inner sep=0pt, outer sep=0pt},
dot/.default = black % size of the circle diameter
}

 % for braces
\usetikzlibrary{decorations.pathreplacing}
% for hashing area
\usetikzlibrary{patterns}
% tableaux var, signe
% source https://www.sqlpac.com/fr/documents/latex-package-tkz-tab-tikz-tableaux-de-signes-et-de-variations-de-fonctions.html
\usepackage{tkz-tab}

\tikzset{
	every node/.style = {font=\Large}
}

\tikzset{
	every axis/.style = {clip=true, grid style = {opacity=.5}}
}
\usepackage{makecell} % commande \thead, dans l'exo 1
\usepackage{eurosym} % pour le symbole euro propre

\begin{document}

\reversemarginpar

\pagestyle{fancy}
\fancyhead[L]{Terminale STMG}
\fancyhead[C]{\textbf{DS n°1 — suites}}
\fancyhead[R]{\today}

\null\vspace{-30pt}
Nom / Prénom : \\

Consignes particulières : 
\begin{itemize}[label=$\bullet$]
	\item 
	La calculatrice est {autorisée}.
	\item 
	L'exercice \ref{exe:1} peut être fait entièrement sur la feuille d'évaluation. Écrire son nom avant de rendre le sujet pour qu'il soit corrigé.
	\item 
	Toute trace de recherche est prise en compte.
%	\item 
%	L'abbréviation $\tq$ signifie ``tel que''.
\end{itemize}

\hrule

\exe{6}{
	Vrai ou faux ? Cocher la case correspondante (aucune justification n'est attendue). \\
	\vspace{-20pt}
	\begin{center}
	\begin{tabular}{c c c}
		\hspace{10cm} & Vrai & Faux \\
		La suite définie par $u_{n+1} = 4 \times u_n$ est définie par récurrence & $\square$ & $\square$  \\
		La suite définie par $u_{n} = 3 + 4n^2$ est une suite arithmétique & $\square$ & $\square$  \\
		\thead{La suite arithmétique de premier terme $u_0 = -1~000$ \\
		et de raison $r= 10^{-3}$ est toujours négative } & $\square$ & $\square$  \\
		La suite définie par $u_n = 10^n$ est une suite géométrique & $\square$ & $\square$ \\
		Une suite géométrique est soit croissante soit décroissante & $\square$ & $\square$ \\
		\thead{Dans un repère orthonormé, les points $(n, u_n), n \in \N$ \\
		d'une suite artithmétique sont alignés} & $\square$ & $\square$ \\
	\end{tabular}
	\end{center}
	\textit{Rappel : une suite est croissante si ses termes sont ``de plus en plus grands'' (ou égaux)}.
}{exe:1}{
Les réponses sont, dans l'ordre : vrai, faux, faux, vrai, faux, vrai.
}
	
\exe{4}{
\begin{enumerate}
\item
On considère une suite arithmétique $(u_n)$ de premier terme $u_0=2$ et de raison $r=7$.
Calculer $u_1$, $u_2$ et $u_{200}$ puis :
\[ \sum_{n=0}^{200} u_n. \]
\item
On considère une suite géométrique $(v_n)$ de premier terme $v_0=3$ et de raison $q=\dfrac14$.
Calculer $v_1$, $v_2$ et $v_{50}$ puis :
\[ \sum_{n=0}^{50} u_{n}. \]
\end{enumerate}
}{exe:2}{
\begin{enumerate}
\item
$u_1=9$, $u_2=16$ et $u_{200}=1~402$
\[ \sum_{n=0}^{200} = 141~102 \]
\item
$v_1=\dfrac34$, $v_2=\dfrac3{16}$ et $v_{50}\approx 2,37 \times 10^{-30}$
\[ \sum_{n=0}^{50} u_{n} \approx 4 \]
\end{enumerate}
}

\exe{3}{
\begin{enumerate}
\item On considère une suite \textbf{arithmétique} $(u_n)$ de raison $r=\dfrac25$ et de troisième terme $u_2 = 5$. Calculer $u_0$.
\item On considère une suite \textbf{géométrique} $(v_n)$ de raison $q=10^{-2}$ et de deuxième terme $v_1 = 2$. Calculer $v_0$.
\end{enumerate}
}{exe:3}{
\begin{enumerate}
\item $u_0=u_2 - 2r = \dfrac{21}5$.
\item $v_0 = \dfrac{v_1}{q} = 200$ .
\end{enumerate}
}

\exe{3}{
\begin{enumerate}
\item On considère une suite \textbf{arithmétique} $(u_n)$ telle que $u_{20} = 100$ et $u_{10} = 50$. Calculer sa raison.
\item On considère une suite \textbf{géométrique} $(v_n)$ telle que $v_7 = 8$ et $v_8=5$. Calculer sa raison.
\end{enumerate}
}{exe:4}{
\begin{enumerate}
\item $u_{20} = u_{10} + 10r$ donc $r=5$.
\item $v_8 = v_7 \times q$ donc $q=\dfrac58$.
\end{enumerate}
}

\exe{4}{
Un ami vous propose de créer une entreprise dont le principe est le suivant :
\begin{itemize}
\item Vous recrutez des ``investisseurs'' en leurs proposant de verser 1~000 \euro~
et en leur promettant qu'ils doubleront leur mise après une semaine. 
\item Pour payer le premier investisseur après une semaine, vous recrutez deux nouveaux investisseurs.
\item Vous poursuivez ainsi, en recrutant toujours deux nouveaux investisseurs pour en payer un.
\end{itemize}
On note $(u_n)$ le nombre de nouveaux investisseurs recrutés en semaine numéro $n$. On suppose que l'entreprise démarre en semaine 0 avec un seul investisseur.
\begin{enumerate}
\item Calculer $u_0$, $u_1$ et $u_2$.
\item Quelle est la nature de la suite $(u_n)$ ? (Arithmétique, géométrique, autre).
\item Estimer le nombre d'investisseurs nécessaires pour que l'entreprise fonctionne sur une année entière (1 an = 52 semaines). 
Le plan de votre ami semble-t-il possible ? Expliquer.
\end{enumerate}
\textit{Ce type de montage se nomme ``système de Ponzi''.}
}{exe:5}{
\begin{enumerate}
\item On ne compte que les \textbf{nouveaux} investisseurs, $u_0=1$, $u_1=2$ et $u_2=8$.
\item C'est une suite géométrique.
\item Le nombre total d'investisseurs est donné par :
\[ \sum_{n=0}^{52} u_n = \dfrac{1-2^{53}}{1-2} \approx 9 \times 10^{15} \]
soit beaucoup plus que l'intégralité de la population terrestre. 
\end{enumerate}
}

%%%%%%%%%%%%

\newpage
\fancyhead[C]{\textbf{Solutions}}
\shipoutAnswer


\end{document}
