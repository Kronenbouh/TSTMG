% Chargement des paquets

\usepackage{amsmath}
\usepackage{amsthm}
\usepackage{amsfonts}
\usepackage{amssymb}
\usepackage{enumerate}
\usepackage{mathtools}
\usepackage{bbm}
\usepackage{xparse, etoolbox}
\usepackage{enumerate}
\usepackage{mathabx}
\usepackage{minted}
\usepackage[french]{babel}
\usepackage{keytheorems}
\usepackage[theorems]{tcolorbox}
\usepackage{hyperref}

% Environnement

%Chargement des paquets
\usepackage{amsmath}
\usepackage{amsthm}
\usepackage{amsfonts}
\usepackage{amssymb}
\usepackage{enumerate}
\usepackage{mathtools}
\usepackage{bbm}
\usepackage{xparse, etoolbox}
\usepackage{enumerate}
\usepackage{mathabx}
\usepackage{minted}
\usepackage[french]{babel}
\usepackage{keytheorems}
\usepackage[theorems]{tcolorbox}
\usepackage{hyperref}
\usepackage[framemethod=tikz]{mdframed}

%Ensembles de nombres
\newcommand{\N} {\mathbb{N}}
\newcommand{\Ne}{\N^\ast}
\newcommand{\Z} {\mathbb{Z}}
\newcommand{\D} {\mathbb{D}}
\newcommand{\Q} {\mathbb{Q}}
\newcommand{\R} {\mathbb{R}}
\newcommand{\Rb}{\overline{\mathbb{R}}}
\newcommand{\Rp}{\R_+}
\newcommand{\Rm}{\R_-}
\newcommand{\K} {\mathbb{K}}
\newcommand{\Cx}{\mathbb{C}}

%Opérateurs
\newcommand{\equi}{\Leftrightarrow}

%Normes
\DeclarePairedDelimiter\abs{\lvert}{\rvert}

%tikz
\usepackage{tikz, pgfplots}
\usetikzlibrary{positioning}
\usetikzlibrary{shapes.geometric}
\usetikzlibrary{positioning}
\usetikzlibrary {angles}
\usepackage{tkz-euclide}

\tikzset{
dot/.style = {circle, fill=#1, minimum size=5pt,
              inner sep=0pt, outer sep=0pt},
dot/.default = black % size of the circle diameter
}

 % for braces
\usetikzlibrary{decorations.pathreplacing}
% for hashing area
\usetikzlibrary{patterns}
% tableaux var, signe
% source https://www.sqlpac.com/fr/documents/latex-package-tkz-tab-tikz-tableaux-de-signes-et-de-variations-de-fonctions.html
\usepackage{tkz-tab}

\tikzset{
	every node/.style = {font=\Large}
}

\tikzset{
	every axis/.style = {clip=true, grid style = {opacity=.5}}
}

%Interface théorème
\renewcommand*{\proofname}{Démonstration}

\theoremstyle{definition}
\newtheorem*{nota}{Notation}

\theoremstyle{definition}
\newtheorem*{conv}{Convention}

\theoremstyle{definition}
\newtheorem{ex}{Exemple}[section]

\theoremstyle{remark}
\newtheorem{rmq}{Remarque}

\theoremstyle{definition}
\newtheorem*{idea}{Idée}

%styles pour théorèmes
\newkeytheoremstyle{tcb-thm}
  {
    headpunct={},
    notebraces={}{},
    noteseparator={ : },
    notefont=\bfseries,
    bodyfont=\slshape,
    tcolorbox={
      arc=0mm,
      colback=blue!5!white,
      colframe=blue!50!black,
      },
  }
  
\newkeytheorem{thm}[
  name=Théorème,
  parent=section,
  style=tcb-thm,
  ]  
  
\newkeytheoremstyle{tcb-prop}
  {
    headpunct={},
    notebraces={}{},
    noteseparator={ : },
    notefont=\bfseries,
    bodyfont=\slshape,
    tcolorbox={
      arc=0mm,
      colback=blue!5!white,
      colframe=blue!75!black,
      },
  }
 
\newkeytheorem{prop}[
  name=Proposition,
  parent=section,
  style=tcb-prop,
  ] 
  
\newkeytheorem{coro}[
  name=Corollaire,
  parent=section,
  style=tcb-prop,
  ]
  
\newkeytheoremstyle{tcb-lem}
  {
    headpunct={},
    notebraces={}{},
    noteseparator={ : },
    notefont=\bfseries,
    bodyfont=\slshape,
    tcolorbox={
      arc=0mm,
      colback=blue!5!white,
      colframe=blue!100!black,
      },
  }
  
\newkeytheorem{lem}[
  name=Lemme,
  parent=section,
  style=tcb-lem,
  ]
 
  
\newkeytheoremstyle{tcb-def}
  {
    headpunct={},
    notebraces={}{},
    noteseparator={ : },
    notefont=\bfseries,
    bodyfont=\slshape,
    tcolorbox={
      arc=0mm,
      colback=orange!5!white,
      colframe=orange!75!black,
      },
  }
  
\newkeytheorem{defn}[
  name=Définition,
  parent=section,
  style=tcb-def,
  ]
 
\newkeytheorem{rapl}[
  name=Rappel,
  style=tcb-def,
  ]
  
\newkeytheoremstyle{tcb-exo}
  {
    headpunct={},
    notebraces={}{},
    noteseparator={ : },
    notefont=\bfseries,
    bodyfont=\slshape,
    tcolorbox={
      arc=0mm,
      colframe=black,
      colback=white,
      },
  }
  
\newkeytheorem{exo}[
  name=Exercice,
  style=tcb-exo,
  ]
  
\surroundwithmdframed[
	hidealllines=true,
	leftline=true,
	innerleftmargin=10pt,
	innerrightmargin=10pt,
	innertopmargin=-4pt,
	nobreak=true,
]{proof}

% Commandes perso

%Ensembles de nombres
\newcommand{\N} {\mathbb{N}}
\newcommand{\Ne}{\N^\ast}
\newcommand{\Z} {\mathbb{Z}}
\newcommand{\D} {\mathbb{D}}
\newcommand{\Q} {\mathbb{Q}}
\newcommand{\R} {\mathbb{R}}
\newcommand{\Rb}{\overline{\mathbb{R}}}
\newcommand{\Rp}{\R_+}
\newcommand{\Rm}{\R_-}
\newcommand{\K} {\mathbb{K}}
\newcommand{\Cx}{\mathbb{C}}

%Opérateurs
\newcommand{\equi}{\Leftrightarrow}

%Normes
\DeclarePairedDelimiter\abs{\lvert}{\rvert}

%Commande d'exo
\newcommand{\exe}[4]{
	\begin{Exercise}[title=#1, label=#3]
		\marginpar{\mbox{\scriptsize(solution p.\pageref{\ExerciseLabel-Answer})}}
		#2
	\end{Exercise}
	\begin{Answer}[ref=#3]
		#4
	\end{Answer}
}

%tikz
\usepackage{tikz, pgfplots}
\usetikzlibrary{positioning}
\usetikzlibrary{shapes.geometric}
\usetikzlibrary{positioning}
\usetikzlibrary {angles}
\usepackage{tkz-euclide}

\tikzset{
dot/.style = {circle, fill=#1, minimum size=5pt,
              inner sep=0pt, outer sep=0pt},
dot/.default = black % size of the circle diameter
}

 % for braces
\usetikzlibrary{decorations.pathreplacing}
% for hashing area
\usetikzlibrary{patterns}
% tableaux var, signe
% source https://www.sqlpac.com/fr/documents/latex-package-tkz-tab-tikz-tableaux-de-signes-et-de-variations-de-fonctions.html
\usepackage{tkz-tab}

\tikzset{
	every node/.style = {font=\Large}
}

\tikzset{
	every axis/.style = {clip=true, grid style = {opacity=.5}}
}
\usepackage{makecell} % commande \thead, dans l'exo 1
\usepackage{eurosym} % pour le symbole euro propre

\begin{document}

\reversemarginpar

\pagestyle{fancy}
\fancyhead[L]{Terminale STMG}
\fancyhead[C]{\textbf{DS n°1 — suites — sujet B}}
\fancyhead[R]{\today}

\null\vspace{-30pt}
Nom / Prénom : \\

Consignes particulières : 
\begin{itemize}[label=$\bullet$]
	\item 
	La calculatrice est {autorisée}.
	\item 
	L'exercice \ref{exe:1} peut être fait entièrement sur la feuille d'évaluation. Écrire son nom avant de rendre le sujet pour qu'il soit corrigé.
	\item 
	Toute trace de recherche est prise en compte.
%	\item 
%	L'abbréviation $\tq$ signifie ``tel que''.
\end{itemize}

\hrule

\exe{6}{
	Vrai ou faux ? Cocher la case correspondante (aucune justification n'est attendue). \\
	\vspace{-20pt}
	\begin{center}
	\begin{tabular}{c c c}
		\hspace{10cm} & Vrai & Faux \\
		La suite définie par $u_n = (5+3)^n$ est une suite géométrique & $\square$ & $\square$ \\
		La suite définie par $u_{n+1} = 3 \times u_n + 2$ est définie par récurrence & $\square$ & $\square$  \\
		La suite définie par $u_{n+1} = u_n \times 4 + 1$ est une suite arithmétique & $\square$ & $\square$  \\
		\thead{La suite arithmétique de premier terme $u_0 = 1~000$ \\
		et de raison $r= 10^{-3}$ est toujours positive } & $\square$ & $\square$  \\
		Une suite géométrique est soit croissante soit décroissante & $\square$ & $\square$ \\
		\thead{Dans un repère orthonormé, les points $(n, u_n), n \in \N$ \\
		d'une suite artithmétique sont alignés} & $\square$ & $\square$ \\
	\end{tabular}
	\end{center}
	\textit{Rappel : une suite est croissante si ses termes sont ``de plus en plus grands'' (ou égaux)}.
}{exe:1}{
Les réponses sont, dans l'ordre : vrai, vrai, faux, faux, faux, vrai.
}
	
\exe{4}{
\begin{enumerate}
\item
On considère une suite arithmétique $(u_n)$ de premier terme $u_0=2$ et de raison $r=\left(\dfrac14 \right )^2$.
Calculer $u_1$, $u_2$ et $u_{200}$ puis :
\[ \sum_{n=0}^{200} u_n . \]
\item
On considère une suite géométrique $(v_n)$ de premier terme $v_0=3$ et de raison $q=5 \times 10^{-1}$.
Calculer $v_1$, $v_2$ et $v_{50}$ puis :
\[ \sum_{n=0}^{50} u_{n} . \]
\end{enumerate}
}{exe:2}
%%{\begin{enumerate}
%%\item
%%$u_1 = 2 + \dfrac1{16} = \dfrac{33}{16}, u_2 =  \dfrac{34}{16} = dfrac{17}8, u_200 = 2 + 200 \times \dfrac{1}{16} = \dfrac{132}{16} = \dfrac{29}{2}$
%%\[ \sum_{n=0}^{200} u_n = \dfrac{201}2 \left ( 2+\dfrac{29}2 \right ) = \dfrac{6633}4. \]
%%\item
%%$v_1 = \dfrac32, v_2 = \dfrac34, v_50 \approx 2,66 \times 10^{-15}$
%%\[ \sum_{n=0}^{50} u_{n} \approx 3 .\]
%%\end{enumerate}
%}


\exe{3}{
\begin{enumerate}
\item On considère une suite \textbf{arithmétique} $(u_n)$ de raison $r=\dfrac27$ et de troisième terme $u_2 = \dfrac13$. Calculer $u_0$.
\item On considère une suite \textbf{géométrique} $(v_n)$ de raison $q=5^{-2}$ et de deuxième terme $v_1 = 7$. Calculer $v_0$.
\end{enumerate}
}{exe:3}{
\begin{enumerate}
\item $u_0 = u_2 - 2 \times r = \dfrac13 - 2 \times \dfrac27 = - \dfrac5{21}$.
\item $v_0 = v_1 \div q = 175$.
\end{enumerate}
}

\exe{3}{
\begin{enumerate}
\item On considère une suite \textbf{arithmétique} $(u_n)$ telle que $u_{20} = 7$ et $u_{7} = \dfrac53$. Calculer sa raison.
\item On considère une suite \textbf{géométrique} $(v_n)$ telle que $v_7 = 8$ et $v_9=5$. Calculer sa raison.
\end{enumerate}
}{exe:4}{
\begin{enumerate}
\item $u_20 = u_7 + 13 r \Rightarrow 13r = 7 - \dfrac53 \Rightarrow r = dfrac{16}{39}$.
\item $v_9 = v_8 \times q^2 \Rightarrow q = \sqrt{\dfrac58}$.
\end{enumerate}
}

\exe{4}{
	On estime que tous les mille an après la mort d'un organisme, le nombre d'atomes de carbone 14 diminue de $11\%$.
	Ajourd'hui, au temps $0$, on mesure $2$ millions d'atomes de carbone 14 dans un organisme donné.

	Répondre aux questions suivantes en arrondissant à \textbf{l'atome près}.
	\begin{enumerate}
		\item Écrire $A_n$, le nombre de millions d'atomes de carbone 14 après $n$ milliers d'années, où $n\in\N$ est un entier naturel.
		De quelle nature est la suite (arithmétique, géométrique, autre) ?
		\item Combien d'atomes de carbone 14 restera-t-il après 3 000 ans ?
		\item Combien d'atomes de carbone 14 restera-t-il après 4 000 ans ? 
		Proposer une estimation du nombre d'atomes de carbone 14 restant après 4 500 ans.
		\item À partir de combien de milliers d'années restera-t-il moins de 100 000 atomes de carbone 14 ?
	\end{enumerate}
\textit{La mesure de carbone 14 est utilisée pour dater des organismes morts comme les fossiles.}
}{exe:5}{
	\begin{enumerate}
		\item $A_n = 2 \cdot 10^6 \cdot (0,89)^n$
		\item $A_3 = 1~409~938$
		\item $A_4 = 1~254~845$. En notant $x$ le nombre d'atomes restant après 4 500 ans, on a forcément :
		\[ 1~409~938 < x < 1~254~845 . \]
		\item $A_26 = 96~642$ soit environ 26 000 ans.
	\end{enumerate}
}

\end{document}